% $Author$
% $Date$
% $Revision$
% $Id$
%=================================================================
\ifx\wholebook\relax\else
% --------------------------------------------
% Lulu:
	\documentclass[a4paper,10pt,twoside]{book}
	\usepackage[
		papersize={6in,9in},
		hmargin={.75in,.75in},
		vmargin={.75in,1in},
		ignoreheadfoot
	]{geometry}
	\input{../common.tex}
	\pagestyle{headings}
	\setboolean{lulu}{true}
% --------------------------------------------
% A4:
%	\documentclass[a4paper,11pt,twoside]{book}
%	\input{../common.tex}
%	\usepackage{a4wide}
% --------------------------------------------
    \graphicspath{{figures/} {../figures/}}
	\begin{document}
	\renewcommand{\nnbb}[2]{} % Disable editorial comments
	\sloppy
\fi
%=================================================================
\chapter{Un r\'esum\'e de la syntaxe}
\label{cha:syntax}

\sd{We should add pragmas.}
\on{Please do so.}

% \sd{It would be good to add link to the chapter where the reader can learn about conditional, exceptions and loops.}
% \on{There are links already.}

\sq, comme la plupart des dialectes modernes, \st adopte une syntaxe proche de celle de \st-80.
La \ind{syntaxe} est con\c{c}ue de telle sorte que le texte d'un programme lu \`{a} voix haute ressemble \`{a} du pidgin:

\begin{code}{}
(Smalltalk includes: Class) ifTrue: [ Transcript show: Class superclass ]
\end{code}

\noindent
La syntaxe de \sq est minimaliste.

Pour l'essentiel con\c{c}ue uniquement pour \emph{envoyer des messages} (\ie des expressions) et  \emph{des d\'{e}clarations de m\'{e}thodes}.
Les expressions sont construites \`{a} partir d'un nombre tr\`{e}s r\'{e}duit de primitives.
Seulement 6 mots cl\'{e}s, et pas de syntaxe pour les structures de contr\^{o}le ni pour les d\'{e}clarations de nouvelles classes.

En revanche, tout ou presque est r\'{e}alisable en envoyant des messages \`{a} des objets.
Par exemple, \`{a} la place de la structure de contr\^{o}le if-then-else, \st envoie des messages comme \ct{ifTrue:} \`{a} des objets \clsind{Bool\\'{e}ens} .
Les nouvelles \mbox{(sous-)classes} sont cr\'{e}\'{e}es en envoyant un message \`{a} leur super classe.

%=================================================================
\section{Les \'{e}l\'{e}ments syntaxiques }

Les expressions sont compos\'{e}es des blocs constructeurs suivants:
(i) six mots-cl\'{e}s r\'{e}serv\'{e}s , ou \emph{pseudo-variables}:
\pvind{self}, \pvind{super}, \pvind{nil}, \pvind{true}, \pvind{false}, and \pvind{thisContext},
(ii) des expressions constantes pour des \emphind{objets litt\'{e}raux} comprenant les nombres, les caract\`{e}res, les chaines de caract\`{e}res, les symboles et les tableaux,
(iii) des d\'{e}clarations de variables,
(iv) des affectations,
(v) des closures ou fermetures lexicales - \ind{closures} en anglais -, et
(vi) des messages.
\seeindex{pseudo-variable}{variable, pseudo}

%:FIGURE -- Squeak Syntax in a Nutshell
\begin{table}\centering
	\begin{tabular}{ll}
		\toprule
		Syntaxe & ce qu'elle repr\'{e}sente \\
		\midrule
		\lct{startPoint}			&	un nom de variable\\
		\lct{Transcript}			&	un nom de variable globale\\
		\lct{self}				&	une pseudo-variable \\
		\midrule
		\lct{1}				 	&	un entier decimal \\
		\lct{2r101}				&	un entier binaire \\
		\lct{1.5}					& un nombre flottant \\
		\lct{2.4e7}				&	une notation exponentielle \\
		\lct{\$a}					& le caract\`{e}re `a' \\
		\lct{'Hello'}				&	la chaine ``Hello'' \\
		\lct{\#Hello}				&	le symbole \lct{\#Hello} \\
		\lct{\#(1 2 3)}			&	un tableau de litt\'{e}raux \\
		\lct{\{1. 2. 1+2\}}		&	un tableau dynamique \\
		\midrule
		\lct{"a comment"} 		&	un commentaire  \\
		\midrule
		\lct{| x y |}				&	une d\'{e}claration de 2 variables \lct{x} et \lct{y}	\\
		\lct{x := 1}				&	affectation de 1 \`{a} \lct{x} \\
		\lct{[ x + y ]}			&	un bloc qui \'{e}value \lct{x+y} \\
		\lct{<primitive: 1>}		&	une primitive de la MV (Machine Virtuelle ou annotation\\
		\midrule
		\lct{3 factorial}			&	un message unaire \\
		\lct{3+4}					&	un message binaire \\
		\lct{2 raisedTo: 6 modulo: 10}		&	un message \`{a} mot-cl\'{e} \\
		\midrule
		\lct{$\uparrow$ true} 			&	retourne la valeur true	\\
		\lct{Transcript show: 'hello'. Transcript cr }		& un s\'{e}parateur d'expression (\lct{.})	\\
		\lct{Transcript show: 'hello'; cr}					
		& un message en cascade (\lct{;}) \\
		\bottomrule
	\end{tabular}
\caption{R\'{e}sum\'{e} de la syntaxe de \sq \label{tab:syntax}}
\end{table}


Dans la \tabref{syntax} nous pouvons voir des exemples divers d'\'{e}l\'{e}ments syntaxiques.
\begin{description}
\item[Variables locales.] \ct{startPoint} est un nom de variable, ou identifiant.
Par convention, les identifiants sont compos\'{e}s de mots au format ``\ind{camelCase}'' (\ie chaque mot except\'{e} le premier d\'{e}bute par une lettre majuscule ).
La premi\`{e}re lettre d'une variable d'instance, d'une m\'{e}thode ou d'un bloc argument, ou d'une variable temporaire doit \^{e}tre en minuscule.
Ce qui indique au lecteur que la port\'{e}e de la variable est priv\'{e}e .

\item[Variables partag\'{e}es]	Les identifiants qui d\'{e}butent par une lettre majuscule sont des \subind{variables} {variables} {globales} , des  \subind{classe} {variables} de {classes}, des \subind{dictionnaire} {dictionnaires} ou des noms de classes.
\ct{Transcript} est une variable globale, une instance de la classe \ct{TranscriptStream}.
\seeindex{variable globale}{variable, globale}
\seeindex{dictionnaire}{variable, pool}
\seeindex{variable!classe}{classe, variable}

\item[Le receveur.] \pvind{self} est un mot-cl\'{e} qui pointe vers l'objet sur lequel la m\'{e}thode courante s'ex\'{e}cute. Nous le nommons  ``le receveur'' car cet objet devra normalement re\c{c}evoir le message qui provoque l'\'{e}x\'{e}cution de la m\'{e}thode.
\self est appel\'{e}e une ``\subind{pseudo} {pseudo-variable}'' puisque nous ne pouvons rien lui affecter.

\item[Les entiers.] En plus des entiers d\'{e}cimaux habituels comme \ct{42}, \sq propose aussi une \ind{notation selon une base num\'{e}rique}.
\ct{2r101} est \ct{101} en base 2 (\ie en binaire), qui est \'{e}gal \`{a} l'entier d\'{e}cimal 5.
\index{litt\'{e}ral}
\index{litt\'{e}ral!nombre}

\item[Les nombres flottants] peuvent \^{e}tre sp\'{e}cifi\'{e}s avec leur \ind{exposant} en base dix: \mbox{\ct{2.4e7}} est $2.4 \times 10^7$.
\index{nombres flottants}

\item[Les caract\`{e}res.] Un signe dollar d\'{e}fini un \subind{caract\`{e}re}{litt\'{e}ral}: \ct{$a}\ignoredollar$ est le litt\'{e}ral pour `a'.
Des instances de caract\`{e}res non-imprimables peuvent \^{e}tre obtenues en envoyant des messages ad hoc \`{a} la classe \clsind{Character} , comme \ct{Character space} \cmindex{Character class}{space} et \ct{Character tab}\cmindex{Character class}{tab}.
		
\item[Les chaines de caract\`{e}res.] les apostrophes sont utilis\'{e}es pour d\'{e}finir un  litt\'{e}ral \subind{litt\'{e}ral}{chaine}.
Si vous d\'{e}sirez une chaine comportant une apostrophe , il suffira de doubler l'apostrophe , comme dans \ct{'G''day'}.

\item[Les symboles] sont comme les chaines de caract\`{e}res , en ce sens qu'ils comportent une suite de caract\`{e}res.  
Mais, contrairement \`{a} une chaine, un \subind{symbole}{litt\'{e}ral} doit \^{e}tre globalement unique.
Il y a seulement un objet symbole \ct{#Hello} mais il peut y avoir plusieurs objets chaines de caract\`{e}res ayant la valeur \ct{'Hello'}.
\seeindex{\#@{\textsf{\#}}}{symbole litt\'{e}ral}

\item[Des tableaux \`{a} la compilation] sont d\'{e}finis par \ct{#( )}, les objets litt\'{e}raux sont s\'{e}par\'{e}s par des espaces.
A l'int\'{e}rieur des parenth\`{e}ses tout doit \^{e}tre constant durant la compilation.
Par exemple,  \ct{#(27 #(true false) abc)} est un \subind{Tableau}{litt\'{e}ral} de trois \'{e}l\'{e}ments: l'entier \ct{27}, le tableau \`{a} la compilation contenant deux bool\'{e}ens, et le symbole \ct{#abc}.

\item[Des tableaux \`{a} l'ex\'{e}cution.] Les accolades \ct|{ }| d\'{e}finissent un (\subind{tableau} {dynamique} \`{a} l'ex\'{e}cution.
Les \'{e}l\'{e}ments sont des expressions separ\'{e}es par des points.
Ainsi \ct|{ 1. 2. 1+2 }| d\'{e}fini un tableau dont les \'{e}l\'{e}ments sont 1, 2, et le r\'{e}sultat de l'\'{e}valuation de 1+2.
(La notation entre accolades est particuli\`{e}re \`{a} \sq un dialecte de \st!
Dans d'autres \st{}s vous devez explicitement construire des tableaux dynamiques.)

\item[Les commentaires] sont encadr\'{e}s par des guillemets.
\ct{"hello"} est un \ind{commentaire}, et non une chaine, et donc est ignor\'{e} par le compilateur de \sq.
Les commentaires peuvent se r\'{e}partir sur plusieurs lignes.
		
\item[Les d\'{e}finitions des variables locales.] Des barres verticales \ct{| |} limitent les \subind{d\'{e}claration} {d\'{e}clarations} d'une ou plusieurs variables locales dans une m\'{e}thode (et aussi dans un bloc).
% \seeindex{\|@{\textsf{\|\|}}}{assignment}
% Can't seem to index or-bars! (special char for index macro)
\seeindex{d\'{e}claration}{d\'{e}claration de variable}

\item[Affectation.]	\ct{:=} affecte un objet \`{a} une variable.
Quelquefois vous verrez \`{a} la place une $\leftarrow$ .
Malheureusement, tant qu'elle ne sera pas un caract\`{e}re \textsc{ascii}, il apparaitra sous la forme d'underscore  \`{a} moins que vous utilisiez une fonte sp\'{e}ciale.
Ainsi, \ct{x := 1} est identique \`{a} \ct{x _ 1} ou \ct{x UNDERSCORE 1}. Vous devrez utiliser  \ct{:=} puisque les autres repr\'{e}sentations ont \'{e}t\'{e} supprim\'{e}es depuis la version 3.9 de Squeak.
\index{affectation}
\seeindex{:=@{\textsf{:=}}}{affectation}
\seeindex{\_@{\textsf{\_}}}{affectation}
\seeindex{<-@{$\leftarrow$}}{affectation}

\item[Blocs.] Des crochets \ct{[ ]} definissent un \ind{bloc}, aussi connu comme une closure  ou une fermeture lexicale, laquelle est un objet \`{a} part enti\`{e}re repr\'{e}sentant une fonction.
Comme nous le verrons, les blocs peuvent avoir des arguments et des variables locales.
\seeindex{[ ]@{\textsf{[ ]}}}{bloc}
\seeindex{closure}{bloc}
\seeindex{fermeture lexicale}{bloc}

\item[Primitives.]	\ct{<primitive: ...>} code l'invocation d'une \ind{primitive} de la VM (\ind{machine virtuelle}
(\ct{<primitive: 1>} est la VM primitive de \ct{SmallInteger>>>+}.).
Tout code suivant la primitive est ex\'{e}cut\'{e} seulement si la primitive \'{e}choue. Depuis la version 3.9 de Squeak, la m\^{e}me syntaxe est aussi employ\'{e} pour des annotations de m\'{e}thode.

\item[Les messages unaires] sont des simples mots (comme \ct{factorial}) envoy\'{e}s \`{a} un receveur (comme \ct{3}).
\index{message!unaire}
\seeindex{message unaire}{message, unaire}

\item[Les messages binaires] sont des op\'{e}rateurs (comme \ct{+}) envoy\'{e}s \`{a} un receveur et ayant un seul argument. Dans \ct{3+4}, le receveur est \ct{3} et l'argument est \ct{4}.
\index{message!binaire}
\seeindex{message binaire}{message, binaire}

\item[Les messages \`{a} mots-cl\'{e}s] sont des mots-cl\'{e}s multiples (comme \ct{raisedTo: modulo:}), chacun se terminant par un deux points (:) et ayant un seul argument. 
Dans l'expression \ct{2 raisedTo: 6 modulo: 10}, le \emphind{message selector} \ct{raisedTo:modulo:} prend les deux arguments \ct{6} et \ct{10}, chacun suivant le : . Nous envoyons le message au receveur \ct{2}.
\index{message!mot-cl\'{e}}
\seeindex{message de mot-cl\'{e}}{message, mot-cl\'{e}}

\item[Le retour d'une m\'{e}thode.] \ct{^} est employ\'{e} pour obtenir le \emphind{retour} d'une m\'{e}thode.  (Il faut taper \verb|^| pour obtenir le caract\`{e}re \ct{^}.)
\md{\ct{^} donne toujours un retour de la m\'{e}thode, m\^{e}me s'il est utilis\'{e} dans un bloc, il donnera le retour de la m\'{e}thode inser\'{e}e dans le bloc.}

\item[Suites d'instructions.]	Un point (\ct{.}) est le \emph{s\'{e}parateur}{ }{d'instructions}. Placer un point entre deux expressions les transforme en deux instructions ind\'{e}pendantes.	
\seeindex{point}{s\'{e}parateur instruction}
	\seeindex{point}{s\`{e}parateur instruction}
	\seeindex{\ct{.}}{s\`{e}parateur instruction}

\item[Cascades.] un point virgule peut \^{e}tre utilis\'{e} pour envoyer une \emphind{cascade} de messages \`{a} un receveur unique. Dans \ct{Transcript show: 'hello'; cr} nous envoyons d'abord le message mot-cl\'{e} \ct{show: 'hello'} au receveur  \ct{Transcript}, et puis nous envoyons au m\^{e}me receveur le message unaire \ct{cr}.
	\seeindex{;}{cascade}

\end{description}

Les classes \ct{Number}, \ct{Character}, \ct{String} et \ct{Boolean} sont d\'{e}crites avec plus de d\'{e}tails dans \charef{basic}.
\on{Blocks are described in \charef{blocks}. (Control flow and Iterators).}

%=================================================================
\section{Les pseudo-variables}

Dans \st, il y a 6 mots-cl\'{e}s r\'{e}serv\'{e}s, ou  \emph{pseudo-variables}:
\pvind{nil}, \pvind{true},  \pvind{false},  \pvind{self}, \pvind{super}, and \pvind{thisContext}.
Ils sont appel\'{e}s \subind{pseudo-variables} {pseudo-variables} car ils sont pr\'{e}d\'{e}finis et ne peuvent pas \^{e}tre l'objet d'une affectation.
\ct{true}, \ct{false}, et \ct{nil} sont des constantes tandis que les valeurs de \ct{self}, \ct{super}, et de \ct{thisContext} varient de fa\c{c}on dynamique lorsque le code est ex\'{e}cut\'{e} 

\ct{true} et \ct{false} sont les uniques instances des classes \clsind{Boolean} \clsind{True} et \clsind{False}.
Voir \charef{basic} pour plus de d\'{e}tails.

\pvind{self} se r\'{e}f\`{e}re toujours au receveur de la m\'{e}thode en cours d'ex\'{e}cution.
\ct{super} se r\'{e}f\`{e}re aussi au receveur de la m\'{e}thode en cours, mais quand vous envoyez un message \`{a} \super, la recherche de m\'{e}thode change de sorte qu'il d\'{e}marre de la super-classe relative \`{a} la classe contenant la m\'{e}thode qui utilise \ct{super}.
Pour plus de d\'{e}tails voir \charef{model}.

\ct{nil} est l'objet non d\'{e}fini.
C'est l'unique instance de la classe \clsind{UndefinedObject}. 
Les variables d'instances (et les variables de classe ) sont initialis\'{e}es \`{a} \ct{nil}.

\ct{thisContext} est une pseudo-variable qui repr\'{e}sente la structure du sommet de la pile d'ex\'{e}cution.
En d'autres termes, il repr\'{e}sente le \clsind{MethodContext} ou le \clsind{BlockContext} en cours d'ex\'{e}cution.
\ct{thisContext} est normalement pas utile \`{a} la plupart des programmeurs, mais il est essentiel pour impl\'{e}menter des outils de d\'{e}veloppement comme le deboggeur et il est aussi utilis\'{e} pour g\'{e}rer les exceptions  et les {ZZZ reprises,poursuites,prolongements ZZZ}.

%=================================================================
\section{Envois de messages}

Il y a trois types de messages dans \sq.
\begin{enumerate}
  \item Les messages \emph{unaires}, messages sans argument.
  \ct{1 factorial} envoie le message  \ct{factorial} \`{a} l'objet \ct{1}.
  \item Les messages \emph{binaires} avec un seul argument.
  	\ct{1 + 2} envoie le message \ct{+} avec l'argument \ct{2} \`{a} l'objet \ct{1}.
  \item Les messages \`{a} \emph{mots-cl\'{e}s} qui comportent un nombre arbitraire d'arguments.
  	\ct{2 raisedTo: 6 modulo: 10} envoie le message comprenant les s\'{e}lecteurs 	\ct{raisedTo:modulo:} et les arguments \ct{6} et \ct{10} vers l'objet \ct{2}.
\end{enumerate}

Les s\'{e}lecteurs des messages unaires sont constitu\'{e}s de caract\`{e}res alphanum\'{e}riques, et d\'{e}butent par une lettre minuscule. 
\index{message!unaire}

Les s\'{e}lecteurs des messages binaires sont constitu\'{e}s par un ou plusieurs caract\`{e}res de l'ensemble suivant:
\index{message!binaire}
\begin{code}{}
+ - / \ * ~ < > = @ % | & ! ? ,
\end{code}
\noindent
% [\~\!\@\%\&\*\-\+\=\\\|\?\/\>\<\,]
\on{Il semble que 3 caract\`{e}res ou plus fonctionnent bien, mais il n'est pas possible d'avoir plus d'un ``-'' dans un s\'{e}lecteur binaire. Sans doute \`{a} cause d'un conflit avec l'analyseur (parser) des nombres n\'{e}gatifs?}
\ab{Bon; $-$ est \'{e}trange .}

Les s\'{e}lecteurs des messages \`{a} mots-cl\'{e}s sont form\'{e}s d'une suite de mots-cl\'{e}s alphanum\'{e}riques qui d\'{e}butent par une lettre minuscule et se terminent par : .
\index{message!mot-cl\'{e}}

Les messages unaires ont la plus haute priorit\'{e}, puis viennent les messages binaires et pour finir les messages \`{a} mots-cl\'{e}s, ainsi :
\begin{code}{@TEST}
2 raisedTo: 1 + 3 factorial --> 128
\end{code}
(D'abord nous envoyons \ct{factorial} \`{a} \ct{3}, puis nous envoyons \ct{+ 6} \`{a} \ct{1}, et pour finir nous envoyons \ct{raisedTo: 7} \`{a} \ct{2}.)  
Rappelons que nous utilisons la notation \lct{\emph{expression}}\ct{-->}\lct{\emph{result}} pour montrer le r\'{e}sultat de l'\'{e}valuation d'une expression. 

Priorit\'{e} mise \`{a} part, l'\'{e}valuation s'effectue strictement de la gauche vers la droite, ainsi 
\begin{code}{@TEST}
1 + 2 * 3 --> 9
\end{code}
et non \ct{7}.
Les parenth\`{e}ses permettent de modifier l'ordre d'une \'{e}valuation:
\begin{code}{@TEST}
1 + (2 * 3) --> 7
\end{code}
Les envois de message peuvent \^{e}tre compos\'{e}s gr\^{a}ce \`{a} des points et des points-virgules. Une suite d'expressions s\'{e}par\'{e}es par des points provoque  l'\'{e}valuation de chaque expression dans la suite comme une \emphind{instruction}, une apr\`{e}s l'autre. 
\index{s\'{e}parateur!instruction}

\begin{code}{}
Transcript cr.
Transcript show: 'hello world'.
Transcript cr
\end{code}

\noindent
Ce code enverra \ct{cr} \`{a} l'objet \glbind{Transcript} , puis enverra  \ct{show: 'hello world'}, et pour finir enverra un nouveau \ct{cr}.

Quand une suite de messages doit \^{e}tre envoy\'{e}e \`{a} un \emph{m\^{e}me} receveur, 
ou pour dire les choses plus succinctement en \emphind{cascade}, le receveur est sp\'{e}cifi\'{e} une seule fois, et la suite des messages est s\'{e}par\'{e}e par des points-virgules:

\begin{code}{}
Transcript cr;
    show: 'hello world';
    cr
\end{code}
Ce qui a pr\'{e}cis\'{e}ment le m\^{e}me effet que l'exemple pr\'{e}c\'{e}dent.


%=================================================================
\section{Syntaxe relative aux m\'{e}thodes}
Bien que les expressions peuvent \^{e}tre \'{e}valu\'{e}es n'importe o\`{u} dans \sq (par exemple, dans un espace de travail (workspace), dans un d\'{e}ebogueur (debugger), ou dans un browser), les m\'{e}thodes sont en principe d\'{e}finies dans une fen\^{e}tre du browser, ou dans d\'{e}bogueur .
(Les m\'{e}thodes peuvent aussi ZZZZZZ \^{e}tre rang\'{e}es sur un p\'{e}riph\'{e}rique externe ZZZZZZZZ, mais ce n'est pas la fa\c{c}on habituelle de programmer en \sq.)

Les programmes sont d\'{e}velopp\'{e}s une m\'{e}thode \`{a} la fois, dans l'environnement d'une classe pr\'{e}cise.
(Une classe est d\'{e}finie en envoyant un message \`{a} une classe existante, en demandant de cr\'{e}er une sous-classe , de sorte qu'il n'y a pas de syntaxe sp\'{e}cifique pour cr\'{e}er une classe.)

Voil\`{a} la m\'{e}thode \mthind{String}{lineCount} dans la classe  \clsind{String}.
(La convention habituelle est de se ref\'{e}rer aux m\'{e}thode comme suit \ct{ClassName>>>methodName}, ainsi nous nommerons cette m\'{e}thod \ct{String>>>lineCount}.)

\needlines{9}
\begin{method}[lineCount]{Line count}
String>>>lineCount
   "Answer the number of lines represented by the receiver,
   where every cr adds one line."
   | cr count |
   cr := Character cr.
   count := 1  min: self size.
   self do:
      [:c | c == cr ifTrue: [count := count + 1]].
   ^ count
\end{method}

Sur le plan de la syntaxe, une m\'{e}thode comporte:
\begin{enumerate}
  \item la structure de la m\'{e}thode, avec le nom (\ie \ct{lineCount}) et tous les arguments (aucun dans cet exemple);
  \item les commentaires (qui peuvent \^{e}tre plac\'{e}s n'importe o\`{u}, mais conventionnellement un doit \^{e}tre plac\'{e} au d\'{e}but afin d'expliquer le but de la m\'{e}thode );
  \item les d\'{e}clarations des variables locales (\ie \ct{cr} et \ct{count}); et
  \item un nombre quelconque d'expressions separat\'{e}es par des points; dans notre exemple il y en a quatre.
\end{enumerate}

L'\'{e}valuation de n'importe quelle expression pr\'{e}c\'{e}d\'{e}e par un \ct{^} (ZZZ taper comme ZZZ \verb|^|) provoquera l'arr\^{e}t de la m\'{e}ethode \`{a} cet endroit, donnant en retour la valeur de cette expression.
Une m\'{e}thode qui se termine sans retourner explicitement une expression 
\ind{retournera} de fa\c{c}on implicite \pvind{self}.
\index{retour!implicite}


Les arguments et les variables locales doivent toujours d\'{e}buter par une lettre minuscule.
Les noms d\'{e}butant par une majuscule sont r\'{e}servés aux variables globales.
Les noms des classes, comme par exemple \ct{Character}, sont tout simplement des variables globales qui ZZZ pointent ZZZ vers l'objet repr\'{e}sentant cette classe.

%=================================================================
\section{La syntaxe des blocs}

Les blocs apportent ZZZ un moyen de diff\'{e}rer ZZZ l'\'{e}valuation d'une expression
Un \ind{bloc} est essentiellement une fonction anonyme. Un bloc est \'{e}valu\'{e} en lui envoyant le message \mthind{BlockClosure}{value}.
Le bloc retourne la valeur de la derni\`{e}re expression de son corps, \`{a} moins qu'il y ait un retour explicite (avec \ct{^}), et dans ce cas il ne retourne aucune valeur.
\seeindex{valeur}{closure}

\begin{code}{@TEST}
[ 1 + 2 ] value --> 3
\end{code}

Les blocs peuvent prendre des param\`{e}tres, chacun doit \^{e}tre d\'{e}clar\'{e}
avec un deux points.
Une barre verticale s\'{e}pare les d\'{e}clarations de param\`{e}tres du corps du bloc.
Pour \'{e}valuer un bloc avec un param\`{e}tre, vous devez lui envoyer le message 
 \mthind{Closure}{value:} avec un argument.
Pour un bloc \`{a} deux param\`{e}tres  \mthind{BlockClosure}{value:value:}, et ainsi de suite, jusqu'\`{a} 4 arguments.

\begin{code}{@TEST}
[ :x | 1 + x ] value: 2 --> 3
[ :x :y | x + y ] value: 1 value: 2 --> 3
\end{code}

Si vous avez un bloc comportant plus de quatre param\`{e}tres, vous devez utiliser
\mthind{Block Closure}{valueWithArguments:} et passer les arguments à l'aide d'un tableau.
(Un bloc comportant un grand nombre de param\`{e}tres est souvent r\'{e}v\'{e}lateur d'un probl\`{e}me au niveau de sa conception.)

Des blocs peuvent aussi d\'{e}clarer des variables locales, lesquelles seront entour\'{e}es par des barres verticales, tout comme des d\'{e}clarations de variables locales dans une m\'{e}thode.
ZZZ Les variables locales ZZZ sont d\'{e}clar\'{e}es apr\`{e}s chaque argument:
\index{variable!declaration}

\begin{code}{@TEST}
[ :x :y | | z | z := x+ y. z ] value: 1 value: 2 --> 3
\end{code}

Les blocs sont en fait des \emph{fermetures} lexicales, d\`{e}s lors qu'ils peuvent se r\'{e}f\'{e}rer \`{a} des variables du contexte ZZZ qu'ils engendrent ZZZ.
Le bloc suivant concerne la variable \ct{x} ZZZ de son environnement ZZZ:

\begin{code}{@TEST}
| x |
x := 1.
[ :y | x + y ] value: 2 --> 3
\end{code}

Les blocs sont des instances de la classe \clsind{BlockContext}.
Cel\`{a} signifie qu'ils sont des objets, de sorte qu'ils peuvent \^{e}tre affect\'{e}s \`{a} des variables et \^{e}tre pass\'{e}s comme arguments \`{a} l'instar de tout autre objet.

% For both understandability and performance, it is better for blocks to refer only to their parameters and local variables; blocks that do not refer external variables are optimized by the compiler.
% MARCUS sez: I would just delete the sentence. There is nothing optimized, accessign outer temps is as fast as inner, so the only reason to avoid accessing outer temps would be that the code is easier to understand. But that's a relatively weak argument, I think.
% However, the ability to refer (``capture'') non-local variables can be very powerful when it is needed. 

\emph{Avertissement:}

Dans l'actuelle version (la 3.9), \sq ne supporte pas en r\'{e}alit\'{e} ZZZ les vraies fermetures lexicales ZZZ, puisque les arguments des blocs sont en fait simulat\'{e}s comme \'{e}tant des  variables temporaires de la m\'{e}thode qu'ils ZZZ contiennent ZZZ . Il existe un nouveau compilateur qui supporte compl\`{e}tement les fermetures lexicales (block closures en anglais), mais il est encore en d\'{e}veloppement et non utilis\'{e} par d\'{e}faut.

Dans quelques situations ce probl\`{e}me peut entrainer des conflits de nommage.

Cette situation se produit car \sq est construit sure une des premi\`{e}res impl\'{e}mentation de \st.
Si vous rencontrez ce probl\`{e}me, examinez ZZZ l'expediteur ZZZ de la m\'{e}thode \mthind{BlockClosure}{fixTemps}, ou chargez le \emphind{Closure Compiler}.

%\paragraph{Really important.} \^\ acts as an escaping mechanism. 
%Return expressions inside a nested block expression will terminate the enclosing method.
%In the example 

%\begin{script}[detect]{...} when the expression \ct{^\ x@y} is executed, the method \ct{detect:}
% escapes the current iteration and returns it. 

%TwoLevelSet>>detect: aBlock

%   firstLevel keysAndValuesDo: [ :x :v |
%      v do: [ :y | (aBlock value: x@y) ifTrue: [^x@y]]
%   ].
%   ^nil
%\end{script}


%=================================================================
\section{Un r\'{e}sumé des Tests et des It\'{e}rations}

\st n'offre aucune syntaxe sp\'{e}cifique ZZZ pour les structures de contr\^{o}le ZZZ.
Typiquement celles-ci sont obtenues par l'envoi de messages \`{a} des bool\'{e}ens, ou des nombres ou des collections, avec pour arguments des blocs.

Les tests sont obtenus par l'envoi de ces messages  \mthind{Boolean}{ifTrue:}, \mthind{Boolean}{ifFalse:} ou \mthind{Boolean}{ifTrue:ifFalse:} au r\'{e}sultat d'une expression bool\'{e}enne . Voir \charef{basic} pour plus de d\'{e}tails sur les bool\'{e}ens.

\begin{code}{}
(17 * 13 > 220)
   ifTrue: [ 'bigger' ]
   ifFalse: [ 'smaller' ] --> 'bigger'
\end{code}
% ON: Not a test.
% My regex approach cannot handle multi-line expressions :-(

Les boucles (ou it\'{e}rations) sont obtenues typiquement par l'envoi de message \`{a} des blocs, ou des entiers, ou des collections.

Comme la condition de sortie d'une boucle peut \^{e}tre \'{e}valu\'{e}e de fa\c{c}on r\'{e}p\'{e}titive, elle se pr\'{e}sentera sous la forme d'un bloc plut\^{o}t que de celle d'une valeur bool\'{e}enne.

Voici pr\'{e}cis\'{e}ment un exemple d'une pro\'{e}cedure it\'{e}rative:
\index{iteration}
\index{iteration|voiraussi{Collection, iteration}}
\seeindex{boucles}{iteration}
\seeindex{enumeration}{iteration}
\seeindex{tests conditions}{iteration}

\begin{code}{@TEST | n |}
n := 1.
[ n < 1000 ] whileTrue: [ n := n*2 ].
n --> 1024
\end{code}
\cmindex{BlockClosure}{whileTrue:}

\noindent
\mthind{BlockClosure}{whileFalse:} reverses the exit condition.
\begin{code}{@TEST | n |}
n := 1.
[ n > 1000 ] whileFalse: [ n := n*2 ].
n --> 1024
\end{code}

\noindent
\mthind{Integer}{timesRepeat:} offre un moyen simple pour impl\'{e}menter un nombre donn\'{e} d'it\'{e}rations:
\begin{code}{@TEST | n |}
n := 1.
10 timesRepeat: [ n := n*2 ].
n --> 1024
\end{code}

Nous pouvons aussi envoyer le message \mthind{Number}{to:do:} \`{a} un nombre lequel alors ZZZZ which then acts as ZZZZ la valeur initiale d'un compteur de boucle. 
Les deux arguments sont la borne sup\'{e}rieure, et un bloc qui prend la valeur courante du compteur de boucle comme argument:

\needlines{4}
\begin{code}{@TEST | n |}
n := 0.
1 to: 10 do: [ :counter | n := n + counter ].
n --> 55
\end{code}

\paragraph{ZZZZ  Iterateurs de haut niveau. ZZZZ}

Les collections comprennent un grand nombre de classes diff\'{e}rentes, dont beaucoup
ZZZZ acceptent ZZZZ le m\^{e}me protocole  
Les messages les plus importants pour it\'{e}rer sur des collections comprennent 
\mthind{Collection}{do:}, \mthind{Collection}{collect:}, \mthind{Collection}{select:}, \mthind{Collection}{reject:}, \mthind{Collection}{detect:} ainsi que  \mthind{Collection}{inject:into:}.
Ces messages d\'{e}finissent des it\'{e}rateurs de haut niveau qui nous permettent d'\'{e}crire du code tr\`{e}s compact.


Un \clsind{Intervalle} est une collection qui d\'{e}finie un it\'{e}rateur sur une suite de nombre depuis un d\'{e}but jusqu'\`{a} une fin.
\ct{1 to: 10} repr\'{e}sente l'intervalle de 1 \`{a} 10.
Comme il s'agit d'une collection, nous pouvons envoyer lui le message \ct{do:}.
L'argument est un bloc qui est \'{e}valu\'{e} pour chaque \'{e}l\'{e}ment de la collection.

\begin{code}{@TEST | n |}
n := 0.
(1 to: 10) do: [ :element | n := n + element ].
n --> 55
\end{code}

\ct{collect:} construit une nouvelle collection de la m\^{e}me taille, en transformant chaque \'{e}l\'{e}ment.
\begin{code}{@TEST}
(1 to: 10) collect: [ :each | each * each ] --> #(1 4 9 16 25 36 49 64 81 100)
\end{code}

\ct{select:} et \ct{reject:} construisent des collections nouvelles, contenant un sous-ensemble d'\'{e}l\'{e}ments satisfaisant (ou non) la condition du bloc bool\'{e}en.
\ct{detect:} retourne le premier \'{e}l\'{e}ment satisfaisant la condition.
Ne perdez pas de vue que les chaines sont aussi des conditions, ainsi pouvez 
aussi it\'{e}rer sur tous les caract\^{e}res.

\begin{code}{@TEST}
'hello there' select: [ :char | char isVowel ] --> 'eoee'
'hello there' reject: [ :char | char isVowel ] --> 'hll thr'
'hello there' detect: [ :char | char isVowel ] --> $e 
\end{code}

Finalement, vous devez garder \`{a} l'esprit que les collections acceptent aussi un style  functionel avec l'op\'{e}rateur \emph{fold} au travers de la m\'{e}thode \ct{inject:into:}.
Cela vous am\`{e}ne \`{a} g\'{e}n\'{e}rer un r\'{e}sultat cumulatif  utilisant une expression qui d\'{e}bute avec ZZZ une valeur initiale ZZZ et injecte chaque \'{e}l\'{e}ment de la collection.
Sommes et produits sont des exemples typiques.
\seeindex{fold}{\ct{Collection>>>inject:into}}

\begin{code}{@TEST}
(1 to: 10) inject: 0 into: [ :sum :each | sum + each ] --> 55
\end{code}

\noindent
Ce qui est \'{e}quivalent \`{a} \ct{0+1+2+3+4+5+6+7+8+9+10}.

Plus de d\'{e}tails sur les collections et les flux dans le \charef{collections} et dans le \charef{streams}.

%=================================================================
\section{Primitives et Pragmas}

Dans \st tout est objet, et tout est produit par l'envoi de messages.
N\'{e}anmoins, \`{a} certains points nous ``touchons le fond`` .
Certains objets peuvent seulement \^{e}tre productif en invoquant la \ind{machine virtuelle} et les \ind{primitive}{}s.

Par exemple, les objets suivants sont tous impl\'{e}ment\'{e}s au titre de primitives:
allocation de la m\'{e}moire (\mthind{Behavior}{new}, \mthind{Behavior}{new:}),
manipulation des bits (\mthind{Integer}{bitAnd:}, \mthind{Integer}{bitOr:}, \mthind{Integer}{bitShift:}),
pointeur et arithm\'{e}tique des entiers (\ct{+}, \ct{-},  \ct{<},  \ct{>}, \ct{*}, \ct{/ }, \ct{=}, \ct{==}...),
et acc\`{e}s aux tableaux (\mthind{Object}{at:}, \mthind{Object}{at:put:}).
\seeindex{new@{\ct{new}}}{\ct{Behavior>>>new}}

Les primitives sont invoqu\'{e}es avec la syntaxe  \ct{<primitive: aNumber>}.
Une m\'{e}thode qui invoque une telle primitive peut aussi embarquer du code \st, qui sera \'{e}valu\'{e}  \emph{seulement} si la primitive est en \'{e}chec.

Examinons le code pour \cmind{SmallInteger}{+}.
Si la primitive \'{e}choue, l'expression \ct{super + aNumber} sera \'{e}valu\'{e}e et retourn\'{e}e.

\needlines{6}
\begin{method}[primitive]{A primitive method}
+ aNumber 
  "Primitive. Add the receiver to the argument and answer with the result
  if it is a SmallInteger. Fail if the argument or the result is not a
  SmallInteger  Essential  No Lookup. See Object documentation whatIsAPrimitive."

  <primitive: 1>
  ^ super + aNumber
\end{method}

%The other use of primitives is to optimize some crucial methods. The idea is that the system could work 
%without the primitive but it would be slow. The following method shows that the method \ct{@} is calling the primitive 18. Here the point creation is clearly expressible in \st therefore the code after the primitive is just the creation of a point illustrating what the primitive is actually doing. Note that such a code will be never called except if the primitive would failed which is extremely rare.  

%\begin{method}[xxx]{xxx}
%Integer>>@ y 
%   "Primitive. Answer a Point whose x value is the receiver and whose y 
%   value is the argument. Optional. No Lookup. See Object documentation 
%   whatIsAPrimitive."

%   <primitive: 18>
%   ^Point x: self y: y
%\end{method}


Depuis la version 3.9 de \sq, la syntaxe avec <....> est aussi utilis\'{e}e pour les annotations de m\'{e}thode que l'on appelle des pragmas.

\sd{we should give an example}\ab{Please do!  Is don't know about these.}

%=================================================================
\section{R\'{e}sum\'{e} du chapitre}

\begin{itemize}

\item	\sq a (seulement) six mots r\'{e}serv\'{e}s aussi appel\'{e}s \textit{pseudo-variables}: \ct{true}, \ct{false}, \ct{nil}, \ct{self}, \ct{super},  and  \ct{thisContext}.

\item	Il y a cinq types d'objets litt\'{e}raux: les nombres (\ct{5}, \ct{2.5}, \ct{1.9e15}, \ct{2r111}), les caract\`{e}res (\ct{$a}), les chaines (\ct{'hello'}), les symboles (\ct{#hello}), et les tableaux (\ct{#('hello' #hello)})

\item	Les chaines sont d\'{e}limit\'{e}es par des apostrophes, et les commentaires par des guillemets. Pour obtenir une apostrophe dans une chaine, il suffit de la doubler.


\item	Contrairement aux chaines, les symboles sont oblig\'{e}s d'\^{e}tre globalement unique.

\item	Employez \ct{#( ... )} pour d\'{e}finir un tableau litt\'{e}ral.
		Employez \ct|{ ... }| pour d\'{e}finir un tableau dynamique.
		Sachez que
		\ct{#( 1 + 2 ) size --> 3}, mais que 
		\ct|{ 1 + 2 } size --> 1|

\item	Il y a trois types de messages:
		\emph{unaire} (\eg \ct{1 asString}, \ct{Array new}),
		\emph{binaire} (\eg \ct{3 + 4}, \ct{'hi' , ' there'}), and
		\emph{\`{a} mots-cl\'{e}s} (\eg \ct{'hi' at: 2 put: $o})

\item	Un envoi de message \emph{en cascade}  est une suite de messages envoy\'{e}s \`{a} la m\^{e}me cible, tous s\'{e}par\'{e}s par des ; :
\ct{OrderedCollection new add: #calvin; add: #hobbes; size --> 2}

\item	Les variables locales sont d\'{e}clar\'{e}es à l'aide de barres verticales.
		Employez \ct{:=} pour les affectations; \ct{_} ou \ct{UNDERSCORE} marche aussi mais est abandonnée depuis la version 3.9 de \sq .
		\ct{|x| x:=1}

\item	Les expressions sont les messages envoy\'{e}s, les cascades et les affectations, et possiblement  regroup\'{e}es avec des parenth\`{e}ses.
		\emph{Les instructions} sont des expressions s\'{e}parat\'{e}es par des points.

\item	Les blocs ou clotures lexicales sont des expressions limit\'{e}es par des crochets.
		Les blocs peuvent prendre des arguments et peuvent contenir des variables locales (temporaires).
		Les expressions du bloc ne sont \'{e}valu\'{e}es que lorsque vous envoyez un message \ct{value...} avec le bon nombre d'arguments.\\
		\ct{[:x | x + 2] value: 4 --> 6}.

\item	Il n'y a pas de syntaxe particuli\`{e}re pour les structures de contr\^{o}le, mais seulement des messages qui sous condition \'{e}valuent des blocs.\\
		\ct{(\st includes: Class) ifTrue: [ Transcript show: Class superclass ]}

\end{itemize}

%=================================================================
\ifx\wholebook\relax\else
\end{document}\fi
%=================================================================
%%% Local Variables:
%%% coding: utf-8
%%% mode: latex
%%% TeX-master: t
%%% TeX-PDF-mode: t
%%% ispell-local-dictionary: "english"
%%% End:


