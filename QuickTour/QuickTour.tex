% $Author$ mathieu chappuis + martial boniou
% $Date$
% $Revision$ fini le: 
%%%%%%%%%%%%%%%%%%%%%%
% note temporaire de Martial destine aux relectures:
% collapse a window --> ranger la fenetre (et pas reduire, si vous
% voyez cette erreur, SVP merci de corriger)
%%%%%%%%%%%%%%%%%%%%%%
%=================================================================
\ifx\wholebook\relax\else
% --------------------------------------------
% Lulu:
	\documentclass[a4paper,10pt,twoside]{book}
	\usepackage[
		papersize={6in,9in},
		hmargin={.75in,.75in},
		vmargin={.75in,1in},
		ignoreheadfoot
	]{geometry}
	\input{../common.tex}
	\pagestyle{headings}
	\setboolean{lulu}{true}
% --------------------------------------------
% A4:
%	\documentclass[a4paper,11pt,twoside]{book}
%	\input{../common.tex}
%	\usepackage{a4wide}
% --------------------------------------------
    \graphicspath{{figures/} {../figures/}}
	\begin{document}
	\renewcommand{\nnbb}[2]{} % Disable editorial comments
	\sloppy
\fi
%=================================================================
\newcommand{\clover}{%
	\raisebox{-0.8ex}[0pt][0pt]{%
		\includegraphics[width=1em]{cloverleafKey}}}
%=================================================================
\chapter{Une visite de \sq}
\label{cha:quick}

%In this chapter we will give you a high-level tour of \sq to help you get comfortable with the environment.
%There will be plenty of opportunities to try things out, so it would be a good idea if you have a computer handy when you read this chapter.

Nous vous proposons dans ce chapitre une premi\`ere visite de \sq afin de vous familiariser avec son environnement.
De nombreux aspects seront abordés; il est conseillé d'avoir une
machine pr\^ete \`a l'emploi pour suivre ce chapitre. 


%We will use this icon: \dothisicon{} to mark places in the text where you should try something out in \sq.
%In particular, you will fire up \sq, learn about the different ways of interacting with the system, and discover some of the basic tools.
%You will also learn how to define a new method, create an object and send it messages.

Cet icône \dothisicon{} dans le texte signalera les étapes où vous devrez essayer quelque chose vous-même.
Vous apprendez à démarrer \sq et les différentes manières d'utiliser l'environnement et les outils de base.
La création des méthodes, des objets et les envois de messages seront également abordés.

%=================================================================
\section{Premiers pas}

%\sq is available as a free \ind{download} from \url{www.squeak.org}.
%There are three parts that you will need to download, consisting of four files (see \figref{download}).

\sq est librement disponible depuis le site principal de \sq: \url{www.squeak.org}.
Vous devez y télécharger 3 archives (pour 4 fichiers principaux qui
constituent une installation courante de \sq; voir \figref{download}) 

\begin{figure}[htb]
\centerline {\includegraphics[width=\textwidth]{annotatedDownload}}
\caption{Téléchargement de \sq. \label{fig:download}}
\end{figure}

\begin{enumerate}

  %\item The \emphind{virtual machine} (VM) is the only part of the system that is different for each operating system and processor.  Pre-compiled virtual machines are available for all the major computing environments.  In \figref{download} we see the VM for the Mac is called \textit{\sq 3.8.15beta1U.app}.

\item La \emphind{machine virtuelle} (agr\'eg\'e en VM pour
  \emph{virtual machine}) est la seule partie de l'environnement qui
  est particulière à chaque système d'exploitation. Des machines
  virtuelles pré-compilées sont disponibles pour la plupart des
  systèmes (Linux, OS/X, Win32). Dans \figref{download}, vous
  remarquerez par exemple la machine virtuelle pour le syst\`eme MacOSX: \textit{\sq 3.8.15beta1U.app}.
%martial: ajout des index dans la vf
\index{machine virtuelle}
\seeindex{VM}{machine virtuelle}

%  \item The \emphind{sources} file contains the source code for all of the parts of \sq that don't change very frequently. In \figref{download} it is called \emph{SqueakV39.sources}. Note that the file SqueakV39.sources is only for versions 3.9 and later of \sq. For earlier versions, use a sources file corresponding to the main version \eg \textit{SqueakV3.sources} for versions of \sq from 3.0 up to 3.8.

  \item Le fichier \emphind{source} contient le code source du système
    \sq qui ne change pas tr\`es souvent. Dans \figref{download}, il
    correspond au fichier \emph{SqueakV39.sources}. Le fichier-source SqueakV39.sources n'est destiné qu'aux versions 3.9 ou supérieures de \sq. Pour des versions antérieures vous devez utiliser un fichier-source (par exemple SqueakV3.sources) de même version que \sq (de 3.0 à 3.8). 
%ajout fr index
\index{fichier!source}
\seeindex{fichier-source}{fichier, source}
\seeindex{SqueakV39.sources}{fichier, source}
\seeindex{SqueakV3.sources}{fichier, source}

%  \item The current \emph{system \ind{image}} is a snapshot of a running \sq system, frozen in time.  It consists of two files: an \emph{.}\emphind{image} file, which contains the state of all of the objects in the system (including classes and methods, since they are objects too), and a \emph{.}\emphind{changes} file, which contains a log of all of the changes to the source code of the system.

\item Le fichier \emph{\ind{image}} est un cliché d'un système en fonctionnement, figé à un instant donné. 
Il est composé de deux fichiers: le premier nommé avec l'extension
\emph{.}\emphind{image} contient l'état de tous les objets du système
ainsi que les classes et les méthodes puisque ce sont aussi des
objets. Le second avec l'extension \emph{.}\emphind{changes} contient
toutes les modifications apportées au code source; elles y sont journalisées.
%ajout fr index
\index{fichier!image}
\index{fichier!changes}

%In \figref{download}, we see that we have grabbed the \textit{Squeak3.9-final-7067} image and sources files.
\end{enumerate}

\dothis{Téléchargez et installez \sq sur votre ordinateur.}
\index{t\'el\'echargement}

%The version of \sq that we have used for developing the examples in this book is \emphind{Squeak-dev}, available from \url{http://damien.cassou.free.fr/squeak-dev}.
%\label{sec:squeakDev}

Dans ce livre, nous avons utilisé pour d\'evelopper tous les exemples
la version \emphind{Squeak-dev} de \sq; cette image est disponible sur
\squeakdev. 
%\url{http://damien.cassou.free.fr/squeak-dev}.
\label{sec:squeakDev}
%We chose this image because it has a wider variety of programming tools pre-installed, and because additional packages can be loaded with a single click.  
Elle contient une large collection d'outils de développement et permet d'installer très facilement des paquetages complémentaires.
%Most of the introductory material in this book will work with any version, so if you already have one installed, you may as well continue to use it.  
Si vous avez déjà une autre version de \sq qui fonctionne sur votre
machine, la plupart des exemples d'introduction de ce livre
fonctionneront. Il n'est donc pas nécessaire de mettre à jour \sq.
%However, if you notice differences between the appearance or behavior of your system and what is described here, do not be surprised. 
D\`es lors, ne soyez pas surpris de constater parfois des différences dans l'apparence ou le comportement que nous décrirons.
%On the other hand, if you are about to download \sq for the first time, you may as well grab the \emph{Squeak-dev} image.
D'un autre c\^ot\'e, si vous téléchargez \sq pour la première fois,
vous devriez rapatrier et utiliser l'image \emph{Squeak-dev}.

%As you work in \sq, the image and changes files are modified, so you need to make sure that they are writable.
Pendant que vous travaillez avec \sq les fichiers \emph{.image} et \emph{.changes} sont modifiés, vous devez vous assurer qu'ils sont accessibles en écriture.
Conservez toujours ces deux fichiers ensemble, \cad dans le même dossier.
Et surtout, ne tentez pas de les modifier avec un éditeur de texte, \sq les utilise pour stocker vos objets de travail et vos changements dans le code source.
Faire une copie de sauvegarde de vos images téléchargées et de vos
fichiers \emph{changes} est une bonne id\'ee; vous pourrez ainsi
toujours démarrer à partir d'une image propre et y recharger votre code.

Les fichiers \emphind{sources} et l'exécutable de la VM peuvent être
en lecture seule\,---\,il est donc possible de les partager entre plusieurs utilisateurs.
Ces quatre fichiers peuvent résider dans le même dossier, mais vous pouvez également placer la machine virtuelle et les fichiers sources dans un dossier partagé distinct.
Vous pouvez adapter l'installation de \sq à vos habitudes de travail
et \`a votre système d'exploitation.

\sd{it would be really nice to have a setup and startup section on PC, Mac and Linux}
\ab{I agree entirely; the reason that this is not here is because I could do only the Mac section.  Damien can probably do Windoze.  Perhaps we can ask on the list for a Linux volunteer?}

% %-----------------------------------------------------------------
% \begin{figure}[htb]
% %\centerline {\includegraphics[width=0.6\textwidth]{download}}
% \centerline {\includegraphics[width=0.95\textwidth]{startup}}
% \caption{A fresh \sq image.\label{fig:startup}}
% \end{figure}

\index{Lancer Squeak}

\paragraph{Lancement.} Pour lancer \sq, selon votre système: glissez
le fichier \emph{.}\emphind{image} sur l'icône de l'exécutable de la
machine virtuelle, ou double-cliquez sur le fichier
\emph{.}\emphind{image}, ou encore, depuis une ligne de commande,
tapez le nom du fichier binaire correspondant \`a la machine virtuelle
suivi du le chemin d'accès au fichier \emph{.}\emphind{image} (si vous
avez installé plusieurs machines virtuelles, le système ne choisira
pas forcément celle qui convient, il sera préférable de
glisser-déposer l'image sur la VM ou d'utiliser la ligne de commande).

Une fois lancé, \sq vous présente une large fenêtre qui contient des espaces de travail (voir \figref{startup}). Notez qu'il n'y a pas de barre de menu, à la place \sq utilise des menus contextuels. 

%martial: j'ai mis 'normalement les reduit' parce que si la barre des
%taches est chargee, la fenetre est minimisee a la OS/2
\dothis{Lancez \sq. Vous pouvez fermer les fen\^etres d\'ej\`a
  ouvertes en cliquant sur l'icône {\sf X} situ\'e sur le coin
  supérieur gauche des fenêtres ou les ranger (ce qui normalement les
  r\'eduit \`a  leur barre de titre) en cliquant sur le symbole \raisebox{-0.2ex}{{\Large $\circ$}} au coin supérieur droit.}

%-----------------------------------------------------------------
%\paragraph{First Interaction.}
\paragraph{Première interaction.}

Les options du menu World (``Monde'' en anglais) pr\'esent\'e dans
\figref{threeButtons:red} sont un bon point de d\'epart.

\dothis{Cliquez à l'aide de la souris dans l'arrière plan de la
  fenêtre principale pour afficher le menu World, puis sélectionnez
  \menu{open \ldots \go workspace} pour créer un nouvel espace de
  travail ou Workspace.}


\begin{figure}[tbh]
	\centering
	\subfigure[Le menu World]{\label{fig:threeButtons:red}%
		\includegraphics[width=0.2\linewidth]{worldMenu}}\hfill
	\subfigure[le menu contextuel]{\label{fig:threeButtons:yellow}%
		\includegraphics[width=0.25\linewidth]{yellowButtonMenuOnWorkspace}}\hfill
	\subfigure[Le halo]{\label{fig:threeButtons:blue}%FIXFR
		\includegraphics[width=0.25\linewidth]{morphicHaloOnWorkspace}}% these braces needed (else no whitespace at end of line)
	\caption{Le menu World (affiché avec le bouton rouge de la
      souris), un menu contextuel (bouton jaune de la souris) et un
      \subind{Morphic}{halo} Morphic (bouton bleu de la souris).\label{fig:threeButtons}}
\end{figure}
\index{bouton rouge}
\index{bouton jaune}
\index{bouton bleu}
%\seeindex{morphic halo}{Morphic}
\seeindex{halo}{Morphic}

% ON: I had to shrink this and move it up to avoid
% it running over the end of the page.
\begin{wrapfigure}[19]{r}{0.25\linewidth}
% The parameters are the number of narrow lines to the right of the figure [19],
% the placement {r} for right, and the width of the figure. Capital R will allow some float.
% Inside the wrapfig environment, linewidth is special --- the width of the figure.
\includegraphics[width=0.95\linewidth]{colouredMouse}
\caption{La souris de l'auteur. Le clic avec la molette correspond au bouton bleu.\label{fig:colouredMouse}}
\end{wrapfigure}

\sq a été conçu à l'origine pour être utilisé avec une souris à trois boutons. Si votre souris en a moins vous pourrez utiliser des touches du clavier en complément de la souris pour simuler les boutons manquants. Une souris à deux boutons fonctionne bien avec \sq, mais si la votre n'a qu'un seul bouton vous devriez songez à adopter un modèle récent avec une molette, qui fera office de troisième bouton : votre travail avec \sq n'en sera que plus agréable.


\sq évite les termes ``clic gauche'' ou ``clic droit'' car leurs effets peuvent varier selon les systèmes ou les réglages utilisateur. \sq désigne les boutons avec des couleurs. Le bouton avec lequel vous obtenez le menu World est intitulé \emphind{bouton rouge}, il est également employé pour selectionner du texte, des choix de menus ou déplacer des fenêtres. Lors de vos premiers pas avec \sq il vous sera sûrement utile d'utiliser cette référence de couleurs avec votre souris, comme montré sur la \figref{colouredMouse}.

Le \emphind{bouton jaune} est l'autre bouton le plus employé dans \sq, vous l'utiliserez pour afficher les menus contextuels qui présentent des options selon le contexte ou plus précisement selon l'endroit et les objets sur lesquels vous cliquez. Voir la \figref{threeButtons:yellow}.


\dothis{Entrez \ct{Time now} dans le workspace.
Puis cliquez avec le bouton jaune dans le workspace.
Et dans le menu qui apparaît sélectionnez \menu{print it}.}

Enfin, le \emphind{bouton bleu} est utilisé pour activer le menu `` \subind{Formes}{halo}''. Ce menu est présenté sous la forme d'une collection de poignées autour de l'objet actif de l'écran et qui permettent d'en changer la taille, de le faire pivoter. Voir \figref{threeButtons:blue}. 
En survolant lentement les poignées avec le pointeur de votre souris, une bulle d'aide en affichera un descriptif.

\dothis{Cliquez avec le bouton bleu sur le Workspace.
Et déplacez la poignée \raisebox{-0.4ex}{\includegraphics[width=1em]{morphicRotate}} (située à proximité du coin inférieur gauche) pour faire pivoter le Workspace.}

Nous recommandons aux personnes gauchères de configurer leur souris et d'affecter le \ind{button rouge} à la gauche de leur souris, le \ind{bouton jaune} à droite et d'utiliser la molette de défilement (si elle est disponible) comme \ind{bouton bleu}.
Avec une souris sans molette il est possible d'invoquer le menu halo en maintenant 
\ct{alt}, \ct{ctrl} ou \ct{option} pendant que vous cliquez sur le \ind{bouton rouge}.
Si vous utilisez un Macintosh avec une souris à un bouton, vous pouvez simuler le second bouton en maintenant la touche \clover{} enfoncée et en cliquant. Si vous prévoyez d'utiliser \sq souvent, nous vous recommandons d'investir dans un modèle à deux boutons.

%  j'ai ajouté CTRL car sur mon linux ni alt ni fn.. ne marchent pour ça. seul ctrl le fait..

Vous pouvez configurer votre souris selon vos souhaits en utilisant les préférences de votre système ou le pilote de votre dispostif de pointage.
\sq vous propose des réglages pour adapter votre souris et les touches spéciales de votre clavier. Vous trouverez le \ind{Preference Browser} dans l'option \menu{open} du menu \menu{World}.
Dans le Preference Browser, la catégorie \menu{general} contient une option \menu{swapMouseButtons} qui permute les boutons jaune et bleu (voir \menu{swapMouseButtons}). Le \menu{keyboard} a une catégorie pour dupliquer les touches de commandes et rendre équivalent une pression sur \ct{alt} à une pression sur \ct{ctrl}.

\dothis{Ouvrez le Preference Browser en cliquant avec le bouton rouge dans l'arrière plan de la fenêtre de \sq et cherchez l'option \menu{swapMouseButtons} en utilisant la zone de recherche.}

\begin{figure}[htb]
\ifluluelse
	{\centerline {\includegraphics[width=\textwidth]{PreferenceBrowser}}}
	{\centerline {\includegraphics[scale=0.7]{PreferenceBrowser}}}
\caption{Le Preference Browser.\label{fig:prefBrowser}}
\end{figure}

%=================================================================
\section{Le menu World}
\index{menu world}

\dothis{Cliquez avec le bouton rouge dans l'arrière plan de \sq.}

Le menu \menu{World} apparaît à nouveau.

La plupart des menus de \sq ne sont pas modaux comme une fenêtre de dialogue pour la sauvegarde d'un fichier que vous devez soit compléter, soit annuler, sans pouvoir travailler en même temps dans l'application. Avec \sq vous pouvez maintenir ces menus sur l'écran en cliquant sur l'icône en forme d'épingle au coin supérieur droit. Essayez! Vous remarquerez que les menus apparaissent  quand  vous cliquez  mais ne disparaissent pas quand vous relâchez votre bouton, ils restent visibles jusqu'à que vous ayez fait une sélection ou cliqué en dehors du menu. Et tous les menus affichés à l'écrans peuvent se déplacer en glissant leur barre de titre, comme n'importe quelle fenêtre.

Le menu World vous offre un moyen d'accéder à la plupart des outils de \sq.

\dothis{Regardez attentivement le menu \menu{world\go{}open \ldots}. }

Vous verrez  les les principaux outils de \sq, et surtout le System Browser (l'un des nombreux navigateurs de classes) et le Workspace. Nous aurons affaire à eux dans les prochains chapitres.

\begin{figure}[htb]
\ifluluelse
	%{\centerline {\includegraphics[width=0.3\textwidth]{OpenMenu}}}
	{\centerline {\includegraphics[scale=0.5]{OpenMenu}}}
	{\centerline {\includegraphics[scale=0.7]{OpenMenu}}}
\caption{L'option  \menu{open \ldots} dans le menu World.\label{fig:openmenu}}
\end{figure}

%=================================================================
\section{Enregistrer, quitter et redémarrer une session \sq.}

\dothis{Affichez le menu World, puis sélectionnez \menu{new morph \ldots} et défilez jusqu'à \menu{from alphabetical list{\go}A-C{\go}BlobMorph}. Vous avez maintenant un Blob (Forme) ``en main'' . Positionnez-le où vous le souhaitez (bouton rouge). Votre forme s'animera.}
\index{Formes}

\begin{figure}[htb]
\begin{minipage}[b]{0.48\textwidth}
\ifluluelse
	{\centerline{\includegraphics[scale=0.5]{Blob}}}
	{\centerline{\includegraphics[scale=0.7]{Blob}}}
	\caption{Une instance d'un \lct{BlobMorph}.\label{fig:blob}}
\end{minipage}
\hfill
\begin{minipage}[b]{0.48\textwidth}
\ifluluelse
	{\centerline {\includegraphics[scale=0.5]{saveAs}}}
	{\centerline {\includegraphics[scale=0.7]{saveAs}}}
	\caption{Le choix de menu \menu{save as \ldots}.\label{fig:saveas}}
\end{minipage}
\end{figure}

\dothis{Sélectionnez \menu{World\go{}save as \ldots} et entrez le nom ``SBE'', puis cliquez sur le bouton \menu{Accept(s)}. Pour finir, sélectionnez \menu{World\go{}save and quit}.}

Le dossier qui contenait les fichiers .image et .changes lorsque vous avez lancé cette session de travail avec \sq contient maintenant deux nouveaux fichiers : ``SBE.\ind{image}'' and ``SBE.\ind{changes}''. Ils contiennent l'image ``vivante'' de votre session \sq au moment qui précédait votre enregistrement avec \menu{save and quit}.
Ces deux fichiers peuvent être copiés à votre convenance dans les dossiers  de votre disque pour être utilisés plus tard. A vous de les invoquer depuis un lien, un glisser-déposer sur la machine virtuelle ou à partir de la ligne de commande.

\dothis{Lancez \sq avec cette image que vous venez de créer : ``SBE.image'' file.}

Vous retrouvez l'état de votre session exactement tel qu'il était avant que vous quittiez \sq. Le Blob est toujours sur votre fenêtre de travail, en train de se déplacer, au pixel près comme vous l'avez abandonné.

%When you start \sq for the first time, the \sq \ind{virtual machine} loads the image file that you provide. This file contains a snapshot of a large number of objects, including a vast amount of pre-existing code and a large number of programming tools (all of which are objects). As you work with \sq, you will send messages to these objects, you will create new objects, and some of these objects will die and their memory will be reclaimed (\ie garbage-collected).

En lançant pour la première fois \sq, la machine virtuelle charge le fichier image que vous spécifiez. Ce fichier contient l'instantané d'un grand nombre d'objets et surtout le code pré-existant accompagné des outils de développement qui sont d'ailleurs des objets comme les autres. En travaillant dans \sq, vous allez envoyer des messages à ces objets, en créer de nouveaux, et certains seront supprimés et l'espace mémoire utilisé sera récupéré. (\ie ramasse-miettes)..

En quittant \sq vous sauvegardez un instantané (image) de tous les objets, les votres et ceux de \sq. En sauvegardant (par ``save''), vous remplacerez l'image courante par l'instantané de votre session. Pour préserver l'image courante, vous devez enregister sous un nouveau nom comme nous venons de le faire.

Chaque fichier \emph{.image} est accompagné d'un fichier \emph{.changes}.
Ce fichier contient un journal de toutes les modifications que vous auriez faites en utilisant l'environnement de développement.
Vous n'avez pas à vous soucier de ce fichier, la plupart du temps.
Mais comme nous allons le voir plus tard, le fichier \emph{.changes} pourra être utilisé pour rétablir votre système \sq à la suite d'erreurs.

L'image sur laquelle vous travaillez provient d'une image de \st-80 créée à la fin des années 1970.
Beaucoup des objets qu'elle contient sont là depuis des décénnies!

Vous pourriez penser que l'utilisation d'une image est incontournable pour stocker et gérer des projets, mais comme nous le verrons bientôt il existe des outils plus adaptés pour gérer le code et travailler en équipe sur des projets.
Les images sont très utiles mais on considère comme une pratique plutôt brusque de les employer pour diffuser et partager vos projets alors qu'il existe des outils tel que Monticello qui proposent de biens meilleurs moyens de suivre les évolutions du code et de le partager entre plusieurs développeurs.


\dothis{Cliquez avec le bouton bleu sur le Blob}

Vous verrez tout autour une collection d'icônes colorées, on les appelle \emph{Handle}(Poignées).  
Cliquez sur la poignée rose qui contient une croix ; le Blob disparaît.

(Pour réussir cette manipulation, vous devrez peut-être faire plusieurs tentatives car le Blob se déplace et peut fuir votre souris et vous empêcher de cliquer à l'endroit attendu.)


\begin{figure}[htb]
\ifluluelse
	{\centerline {\includegraphics[width=\textwidth]{Tools}}}
	{\centerline {\includegraphics[width=0.8\textwidth]{Tools}}}
\caption{The \sq \toolsflap.\label{fig:tools}}
\end{figure}

%=================================================================
\section{Les Workspaces et les fenêtres Transcripts}
\label{sec:transcript}

\dothis{Fermez toutes fenêtres. Cliquez sur l'onglet \menu{Tools} à la droite de la fenêtre principale de \sq pour ouvrir le volet des outils(Tools Flap).}

L'onglet s'élagira et présentera les icônes de certains outils importants de \sq (\figref{tools}). Glissez alors l'icône \ind{Transcript} puis l'icône Workspace.

\dothis{Positionnez et redimensionnez le Transcript et le Workspace pour que dernier recouvre le Transcript.}

Vous pouvez redimensionner les fenêtres en glissant l'un de leurs coins, ou avec le bouton bleu qui affiche les poignées. Utilisez alors l'icône jaune située en bas à droite.

Une seule fenêtre est active à la fois, son titre est alors affiché en gras. Et surtout le pointeur de la souris doit être dans la fenêtre si vous devez entrer du texte.


Le Transcript est un objet qui est couramment utilisé pour afficher des messages du système.
C'est un genre de ``console''.
L'affichage dans la fenêtre Transcript est très lent, si vous la conservez ouverte et que vous affichez le résultat de certaines opérations celles-ci peuvent ralentir plus de 10 fois.
De plus la fenêtre Transcript n'est pas conçue pour recevoir simultanément des messages à afficher provenant de plusieurs objets. %protegé contre acces concurrents

%Workspaces are useful for typing snippets of \st code that you would like to experiment with.
%You can also use workspaces simply for typing arbitrarily text that you would like to remember, such as to-do lists or instructions for anyone who will use your image.
%Workspaces are often used to hold documentation about a captured image, as is the case with the standard image that we downloaded earlier (see \figref{startup}).



\dothis{Type the following text into the workspace:}
\begin{code}{}
Transcript show: 'hello world'; cr.
\end{code}

%\begin{figure}[htb]
%\centerline {\includegraphics[width=0.6\textwidth]{HelloWorld}}
%\caption{Overlapping windows. The workspace is active.\label{fig:helloworld}}
%\end{figure}

Try double-clicking in the workspace at various points in the text you have just typed.
Notice how an entire word, entire string, or the whole text is selected, depending on where you click.

\dothis{Select the text you have typed and yellow-click.
Select \menu{do it (d)}.}
Notice how the text ``hello world'' appears in the transcript window
(\figref{doit}).
Do it again.
(The \menu{(d)} in the menu item \menu{do it (d)} tells you that the keyboard shortcut to \emph{do it} is \short{d}. More on this in the next section!)

\begin{figure}[htb]
\ifluluelse
	{\centerline {\includegraphics[width=\textwidth]{Doit}}}
	{\centerline {\includegraphics[scale=0.7]{Doit}}}
\caption{``Doing'' an expression\label{fig:doit}}
\end{figure}

You have just evaluated your first \st expression!
You just sent the message \ct{show: 'hello world'} to the \ct{Transcript} object, followed by the message \ct{cr} (carriage return).
The \ct{Transcript} then decided what to do with this message, that is, it looked up its \emph{methods} for handling \ct{show:} and \ct{cr} messages and reacted appropriately.

If you talk to Smalltalkers for a while, you will quickly notice that they generally do not use expressions like ``call an operation'' or ``invoke a method'', but instead they will say ``send a message''.
This reflects the idea that objects are responsible for their own actions. 
You never \emph{tell} an object what to do\,---\,instead you politely \emph{ask} it to do something by sending it a message. 
The object, not you, selects the appropriate method for responding to your message.


%=================================================================
\section{Keyboard shortcuts}

If you want  to evaluate an expression, you do not always have to bring up the yellow-button menu. Instead, you can use \ind{keyboard shortcuts}. These are the parenthesized expressions in the menu.  Depending on your platform, you may have to press one of the modifier keys (control, alt, command, or meta).
(We will indicate these generically as \short{\emph{key}}.)

\dothis{Evaluate the expression in the workspace again, but using the keyboard shortcut: \short{d}.}
\index{keyboard shortcut!do it}

In addition to \menu{do it}, you will have noticed \menu{print it}, \menu{inspect it} and \menu{explore it}. Let's have a quick look at each of these.

\dothis{Type the expression \ct{3 + 4} into the workspace. Now \menu{do it} with the keyboard shortcut.}

Do not be surprised if you saw nothing happen! What you just did is send the message \ct{+} with argument \ct{4} to the number \ct{3}. Normally the result \ct{7} will have been computed and returned to you, but since the workspace did not know what to do with this answer, it simply threw the answer away.  If you want to see the result, you should \menu{print it} instead. \menu{print it} actually compiles the expression, executes it, sends the message \ct{printString} to the result, and displays the resulting string.

\dothis{Select \ct{3+4} and \menu{print it} (\short{p}).}
This time we see the result we expect (\figref{printit}).
\index{keyboard shortcut!print it}

\begin{figure}[htb]
% \centerline {\includegraphics[width=0.4\textwidth]{PrintIt}}
\centerline {\includegraphics[scale=0.7]{PrintIt}}
\caption{``Print it'' rather than ``do it''. \label{fig:printit}}
\end{figure}

\needlines{3}
\begin{code}{@TEST}
3 + 4 --> 7
\end{code}
\noindent
We use the notation \ct{-->} as a convention in this book to indicate that a particular \sq expression yields a given result when you \menu{print it}.

\dothis{Delete the highlighted text ``\ct{7}'' (\sq should have selected it for you, so you can just press the delete key). Select \ct{3+4} again and this time \menu{inspect it} (\short{i}).}
\noindent
Now you should see a new window, called an \emphind{inspector}, with the heading \ct{SmallInteger: 7} (\figref{inspectit}).
The inspector is an extremely useful tool that will allow you to browse and interact with any object in the system.
The title tells us that \ct{7} is an instance of the class \clsind{SmallInteger}.
The left panel allows us to browse the instance variables of an object, the values of which are shown in the right panel.
The bottom panel can be used to write expressions to send messages to the object.

\begin{figure}[htb]
\centerline {\includegraphics[scale=0.7]{InspectIt}}
\caption{Inspecting an object. \label{fig:inspectit}}
\end{figure}

\dothis{Type \ct{self squared} in the bottom panel of the inspector on \ct{7} and \menu{print it}.}

\needlines{2}
\dothis{Close the inspector. Type the expression \ct{Object} in a workspace and this time \menu{explore it} (\short{I}, uppercased i).}
\index{keyboard shortcut!explore it}
\index{explorer}

This time you should see a window labelled \clsind{Object} containing the text
\mbox{$\triangleright$ \ct{root: Object}}.
Click on the triangle to open it up (\figref{exploreit}).

\begin{figure}[htb]
\centerline {\includegraphics[scale=0.7]{ExploreIt}}
\caption{Exploring an object. \label{fig:exploreit}}
\end{figure}

The explorer is similar to the inspector, but it offers a tree view of a complex object.
In this case the object we are looking at is the \ct{Object} class.
We can see directly all the information stored in this class, and we can easily navigate to all its parts.

%=================================================================
\section{\sqmap}
\index{SqueakMap}

\sqmap is a web-based catalog of ``packages''\,---\,applications and libraries\,---\,that you can download to your image.
The \ind{package}{}s are hosted in many different places in the world and maintained by many different people. Some of them may only work with a specific version of \sq.
\lr{Maybe mention Package Universes (SqueakMap is not maintained anymore)}

\dothis{Open \menu{World \go open... \go \sqmap package loader}.}
You will need an Internet connection for this to work.  After some time, the \sqmap loader window should appear (\figref{sokoban}).
On the left side is a very long list of packages.
The field in the top-left corner is a search pane that can be used to find what you want in the list.
Type ``\ind{Sokoban}'' in the search pane and hit the return key.
Clicking on the right-pointing triangle by the name of a package reveals a list of available versions. When a package or a version is selected, information about it appears in the right-hand pane.
Navigate to the latest version of \ct{Sokoban}.
With the mouse in the list pane, use the yellow-button menu to \menu{install} the selected package.
(If \sq complains that it is not sure this version of the game will work in your image, just say ``yes'' and go ahead.)
Notice that once a package has been installed, it is marked with an asterisk in the list in the \sqmap package loader. 

\begin{figure}[htb]
\ifluluelse
	{\centerline {\includegraphics[width=\textwidth]{SqueakMap}}}
	{\centerline {\includegraphics[scale=0.7]{SqueakMap}}}
\caption{Using \sqmap to install the Sokoban game.\label{fig:sokoban}}
\end{figure}

\dothis{After installing this package, start up \ct{Sokoban} by evaluating \ct{SokobanMorph random openInWorld} in a workspace.}

% You can also try the \ct{NsGame}; execute it using \ct{NsGame new openInWorld}.
% ON: I could not find NsGame anywhere!

The bottom-left pane in the \sqmap package loader provides various ways to filter the list of packages.   You can choose to see only those packages that are compatible with a particular version of \sq, or only games, and so on.

%=================================================================
\section{The System Browser}

The \ind{system browser} is one of the key tools used for programming.
As we shall see, there are several interesting browsers available for \sq, but this is the basic one you will find in any image.
\seeindex{class browser}{system browser}
\seeindex{browser}{system browser}

\dothis{Open a browser by selecting \menu{World \go open \ldots \go class browser}, or by dragging a Browser from the \toolsflapind.}

\begin{figure}[htb]
\ifluluelse
	{\centerline {\includegraphics[width=\textwidth]{ClassBrowser1}}}
	{\centerline {\includegraphics[scale=0.7]{ClassBrowser1}}}
\caption{The system browser showing the \ct{printString} method of class object.
\label{fig:classBrowser}}
\end{figure}

We can see a system browser in \figref{classBrowser}.
The title bar indicates that we are browsing the class \clsind{Object}.

When the browser first opens, all panes are empty but the leftmost one.
This first pane lists all known \emph{system categories}, which are groups of related classes.
\index{category}

\dothis{Click on the category \scatind{Kernel-Objects}.}
This causes the second pane to show a list of all of the classes in the selected category.

\dothis{Select the class \clsind{Object}.}
Now the remaining two panes will be filled with text.
The third pane displays the \emph{protocols} of the currently selected class.
These are convenient groupings of related methods.
If no \ind{protocol} is selected you should see all methods in the fourth pane.

\dothis{Select the \protind{printing} protocol.}
You may have to scroll down to find it.
Now you will see in the fourth pane only methods related to printing.

\dothis{Select the \mthind{Object}{printString} method.}
Now we see in the bottom pane the source code of the \ct{printString} method, shared by all objects in the system (except those that override it).

%=================================================================
\section{Finding classes}

There are several ways to find a class in \sq.  The first, as we have just seen above, is to know (or guess) what category it is in, and to navigate to it using the browser.
\index{system browser}
\seeindex{system browser!finding classes}{class, finding}
\index{class!finding}

A second way is to send the \ct{browse} message to the class, asking it to open a browser on itself.  Suppose we want to browse the class \clsind{Boolean}.

\dothis{Type \ct{Boolean browse} into a workspace and \menu{do it}.}
A browser will open on the Boolean class (\figref{browseBoolean}).
There is also a \ind{keyboard shortcut} \short{b} (browse) that you can use in any tool where you find a class name;
\index{keyboard shortcut!browse it}
select the name and type \short{b}.

\dothis{Use the keyboard shortcut to browse the class \ct{Boolean}.}

\begin{figure}[hbt]
\ifluluelse
	{\centerline {\includegraphics[width=\textwidth]{Kernel-objects-boolean}}}
	{\centerline {\includegraphics[scale=0.7]{Kernel-objects-boolean}}}
\caption{The system browser showing the definition of class Boolean.
\label{fig:browseBoolean}}
\end{figure}

Notice that when the \ct{Boolean} class is selected but no protocol or method is selected, two panes rather than one appear below the four panes at the top
(\figref{browseBoolean}).
The upper one contains the \emph{class definition}.
This is nothing more than an ordinary \st message that is sent to the parent class, asking it to create a subclass.
Here we see that the class \ct{Object} is being asked to create a subclass named \ct{Boolean} with no instance variables, class variables or ``pool dictionaries'', and to put the class \ct{Boolean} in the \scatind{Kernel-Objects} category.

The lower pane shows the \emph{class comment} --- a piece of plain text describing the class.
If you click on the \button{?} at the bottom of the class pane, you can see the class \subind{class}{comment} in a dedicated pane.

\ab{I thought that this was supposed to be a \emph{Quick} tour!  And here we are describing a tool that I have used maybe twice in 10 years!   In any case, this description should be deferred to the \textbf{Environment} chapter}
\on{I don't see why.  I use the hierarchy browser a lot!  I think it is really useful to know from the beginning, to help you find your through the hierarchy.}
If you would like to explore \sq's inheritance hierarchy, the \emphind{hierarchy browser} can help you.  
This can be useful if you are looking for an unknown subclass or superclass of a known class.
The hierarchy browser is like the system browser, except that the list of classes is arranged as an indented tree mirroring the inheritance hierarchy.

\dothis{Click on \button{hierarchy} in the browser while the class \ct{Boolean} is selected.}
\noindent
This will open a hierarchy browser showing the superclasses and subclasses of \clsind{Boolean}.
% (\figref{booleanhierarchybrowser}).
Navigate to the immediate superclass and subclasses of \ct{Boolean}.

Often, the fastest way to find a class is to search for it by name.  For example, suppose that you are looking for some unknown class that represents dates and times.

\dothis{Put the mouse in the system category pane of the system browser and type \short{f}, or select \menu{find class \ldots (f)} from the yellow-button menu.  Type ``time'' in the dialog box and accept it.} 
\noindent
You will be presented with a list of classes whose names contain ``time'' (see \figref{findit}).  Choose one, say, \ct{Time}, and the browser will show it, along with a class comment that suggests other classes that might be useful.  If you want to browse one of the others, select its name (in any text pane), and type \short{b}.
\index{keyboard shortcut!find ...}
\index{keyboard shortcut!browse it}

\begin{figure}[hbt]
\centerline{
\ifluluelse{
	\includegraphics[width=0.5\textwidth]{FindIt}
	\hspace{1cm}
	\includegraphics[width=0.4\textwidth]{TimeClasses}
}{
	\includegraphics[width=0.4\textwidth]{FindIt}
	\hspace{1cm}
	\includegraphics[width=0.3\textwidth]{TimeClasses}
}
}
\caption{Searching for a class by name.
\label{fig:findit}}
\end{figure}

Note that if you type the complete (and correctly capitalized) name of a class in the find dialog, the browser will go directly to that class without showing you the list of options.

%=================================================================
\section{Finding methods}
\label{sec:quick:methodFinder}

Sometimes you can guess the name of a method, or at least part of the name of a method, more easily than the name of a class.  For example, if you are interested in the current time, you might expect that there would be a method called ``now'', or containing ``now'' as a substring.   But where might it be?
The \emphind{method finder} can help you.
\seeindex{system browser!finding methods}{method, finding}
\index{method!finding}

\dothis{Drag the \menu{method finder} icon out of the  \toolsflapind.
Type ``now'' in the top left pane, and \menu{accept} it (or just press the \textsc{return} key).}
The method finder will display a list of all the method names that contain the substring ``now''.  
To scroll to \ct{now} itself, type ``\ct{n}''; this trick works in all scrolling windows.  Select ``now'' and the right-hand pane shows you the three classes that define a method with this name, as shown in \figref{MethodFinder}.  Selecting any one of them will open a browser on it.

\begin{figure}[hbt]
\centerline {\includegraphics[scale=0.7]{methodFinder-now}}
\caption{The method finder showing three classes that define a method named \ct{now}.
\label{fig:MethodFinder}}
\end{figure}

At other times you may have a good idea that a method exists, but will have no idea what it might be called.
The method finder can still help!  For example, suppose that you would like to find a method that turns a string into upper case, for example, it would translate \ct{'eureka'} into \ct{'EUREKA'}.

\dothis{Type \ct{'eureka' . 'EUREKA'} into the method finder, as shown in \figref{methodFinder-example1}.}
\noindent
The method finder will suggest a method that does what you want.

A star at the beginning of a line in the right pane of the method finder indicates that this method is the one that was actually used to obtain the requested result. 
So, the star in front of \ct{String asUppercase} lets us know that the method \mthind{String}{asUppercase} defined on the class \clsind{String} was executed and returned the result we wanted. The methods that do not have a star are just the other methods that have the same name as the ones that returned the expected result. So \cmind{Character}{asUppercase} was not executed on our example, because \ct{'eureka'} is not a \clsind{Character} object.

\begin{figure}[hbt]
\ifluluelse
	{\centerline {\includegraphics[width=0.9\textwidth]{MethodFinder-example1}}}
	{\centerline {\includegraphics[scale=0.7]{MethodFinder-example1}}}
\caption{Finding a method by example.
\label{fig:methodFinder-example1}}
\end{figure}

You can also use the method finder for methods with arguments; for example, if you are looking for a method that will find the greatest common factor of two integers, you might try \ct{25. 35. 5} as an example.  You can also give the method finder multiple examples to narrow the search space; the help text in the bottom pane explains more.

%=================================================================
\section{Defining a new method}

The advent of \ind{Test Driven Development}\cite{Beck03a} has changed the way that we write code.  
The idea behind Test Driven Development, also called TDD or Behavior Driven Development, is that we write a test that defines the desired behavior of our code \emph{before} we write the code itself.
Only then do we write the code that satisfies the test.
\seeindex{Behavior Driven Development}{Test Driven Development}
\orla{describe the technique where we write a test hat ... subsequently we write ...}

Suppose that our assignment is to write a method that ``says something loudly and with emphasis''.  What exactly could that mean?  What would be a good name for such a method?  How can we make sure  that programmers who may have to maintain our method in the future have an unambiguous description of what it should do?   We can answer all of these questions by giving an example:

\begin{quote}
When we send the message \ct{shout} to the string ``Don't panic'' the result should be ``DON'T PANIC!''.
\end{quote}

\noindent
To make this example into something that the system can use, we turn it into a test method:
\index{testing}
\index{SUnit}

\needlines{3}
\begin{method}[testShout]{A test for a shout method}
testShout
	self assert: ('Don''t panic' shout = 'DON''T PANICBANG')
\end{method} % BANG is the escape for !

How do we create a new method in \sq?   First, we have to decide which class the method should belong to.
In this case, the \ct{shout} method that we are testing will go in class \clsind{String}, so the corresponding test will, by convention, go in a class called \clsind{StringTest}.

\begin{figure}[hbt]
\ifluluelse
	{\centerline {\includegraphics[width=\textwidth]{StringTest-newMethodTemplate}}}
	{\centerline {\includegraphics[width=0.7\textwidth]{StringTest-newMethodTemplate}}}
\caption{The new method template in class \ct{StringTest}.
\label{fig:newMethodTemplate}}
\end{figure}

\dothis{Open a browser on the class \ct{StringTest}, and select an appropriate protocol for our method, in this case \menu{tests - converting}, as shown in \figref{newMethodTemplate}.
The highlighted text in the bottom pane is a template that reminds you what a \st method looks like.
Delete this and enter the code from  \mthref{testShout}.}
Once you have typed the text into the browser, notice that the bottom pane is outlined in red.  This is a reminder that the pane contains unsaved changes.
So select \menu{accept (s)} from the yellow-button menu in the bottom pane, or just type \short{s}, to compile and save your method.
\index{keyboard shortcuts}
\index{keyboard shortcut!accept}
\seeindex{accept it}{keyboard shortcut, accept}

Because there is as yet no method called \ct{shout}, the browser will ask you to confirm that this is the name that you really want\,---\,and it will suggest some other names that you might have intended (\figref{testShoutConfirm}).
This can be quite useful if you have merely made a typing mistake, but in this case, we really \emph{do} mean \ct{shout}, since that is the method we are about to create, so we have to confirm this by selecting the first option from the menu of choices, as shown in \figref{testShoutConfirm}. 

\begin{figure}[hbt]
\ifluluelse
	{\centerline {\includegraphics[width=\textwidth]{testShoutConfirm}}}
	{\centerline {\includegraphics[scale=0.7]{testShoutConfirm}}}
\caption{Accepting the testShout method class \ct{StringTest}.
\label{fig:testShoutConfirm}}
\end{figure}

\dothis{Run your newly created test: open the \ind{SUnit} \emphind{TestRunner}, either by dragging it from the \toolsflapind, or by selecting \menu{World \go open... \go Test Runner}.}

The leftmost two panes are a bit like the top panes in the system browser.  The left pane contains a list of system categories, but it's restricted to those categories that contain test classes.

\dothis{Select \scat{CollectionsTests-Text} and the pane to the right will show all of the test classes in that category, which includes the class \ct{StringTest}.  The names of the classes are already selected, so click \menu{Run Selected} to run all these tests.}

\begin{figure}[hbt]
\ifluluelse
	{\centerline {\includegraphics[width=\textwidth]{testRunnerOnStringTest}}}
	{\centerline {\includegraphics[scale=0.5]{testRunnerOnStringTest}}}
\caption{Running the String tests.
\label{fig:testRunnerTestShout}}
\end{figure}

You should see a message like that shown in \figref{testRunnerTestShout}, which indicates that there was an error in running the tests.  The list of tests that gave rise to errors is shown in the bottom right pane; as you can see, \ct{StringTest>>>#testShout} is the culprit.
(Note that \ct{StringTest>>#testShout} is the Smalltalk way of identifying the \mthind{StringTest}{testShout} method of the \ct{StringTest} class.)
If you click on that line of text, the erroneous test will run again, this time in such a way that you see the error happen: ``\ct{MessageNotUnderstood: ByteString>>>shout}''.
\seeindex{\ct{>>}}{Behavior, \ct{>>}}
\cmindex{Behavior}{>>}

The window that opens with the error message is the \st debugger (see \figref{predebugger}).
\ab{Well, it's actually the \emph{pre-}debugger.  Does this matter?}\damien{I don't think it's important at this point.}
We will look at the \ind{debugger} and how to use it in \charef{env}.

\begin{figure}[hbt]
\ifluluelse
	{\centerline {\includegraphics[width=\textwidth]{Predebugger}}}
	{\centerline {\includegraphics[scale=0.7]{Predebugger}}}
\caption{The (pre-)debugger.}
\label{fig:predebugger}
\end{figure}

The error is, of course, exactly what we expected:  running the test generates an error because we haven't yet written a method that tells strings how to \ct{shout}.  
Nevertheless, it's good practice to make sure that the test fails because this confirms that we have set up the testing machinery correctly and that the new test is actually being run.
Once you have seen the error, you can \button{Abandon} the running test, which will close the debugger window.
Note that often with Smalltalk you can define the missing method using the \button{Create} button, edit the newly-created method in the debugger, and then \button{Proceed} with the test.

Now let's define the method that will make the test succeed!

\dothis{Select class \clsind{String} in the system browser, select the \menu{converting} protocol, type the text in \mthref{shout} over the method creation template, and \menu{accept} it.
(Note: to get a \mbox{\ct{^}}, type \caret). }
\begin{method}[shout]{The shout method}
shout
	^ self asUppercase, 'BANG'
\end{method}

The comma is the string concatenation operation, so the body of this method appends an exclamation mark to an upper-case version of whatever \ct{String} object the \ct{shout} message was sent to.
The $\uparrow$ tells \sq that the expression that follows is the answer to be returned from the method, in this case the new concatenated string.
\seeindex{comma}{Collection, comma operator}
\index{Collection!comma operator}

Does this method work?  Let's run the tests and see.

\dothis{Click on \menu{Run Selected} again in the test runner, and this time you should see a green bar and text indicating that all of the tests ran with no failures and no errors.}
When you get to a green bar\footnotemark, it's a good idea to save your work and take a break.  
So do that right now!
\footnotetext{Actually, you might not get a green bar since some images contains tests for bugs that need to be fixed.
Don't worry about this.
\Squeak is constantly evolving.
%:PROBLEM --- StringTest has a broken test!
% ON: There is a broken test in \ct{StringTest>>>\#testIndexOf}!
}

\begin{figure}[hbt]
\ifluluelse
	{\centerline{\includegraphics[width=\textwidth]{String-Shout}}}
	{\centerline{\includegraphics[scale=0.7]{String-Shout}}}
\caption{The \ct{shout} method defined on class \ct{String}.
\label{fig:String-shout}}
\end{figure}

%=================================================================
\section{Chapter summary}
This chapter has introduced you to the \sq environment and shown you how to use some of the major tools, such as the system browser, the method finder, and the test runner.   You have also seen a little of \sq's syntax, even though you may not understand it all yet.

\begin{itemize}
  \item A running \sq system consists of a \emph{virtual machine}, a \emph{sources} file, and \emph{image} and \emph{changes} files. Only these last two change, as they record a snapshot of the running system.
  \item When you restore a \sq image, you will find yourself in exactly the same state\,---\,with the same running objects\,---\,that you had when you last saved that image.
  \item \sq is designed to work with a three-button mouse. The buttons are known as the \emph{red}, the \emph{yellow} and the \emph{blue} buttons. If you don't have a three-button mouse, you can use modifier keys to obtain the same effect.
  \item You use the \ind{red button} on the \sq background to bring up the \emph{World menu} and launch various tools. You can also launch tools from the \toolsflapind at the right of the \sq screen.
  \item A \emph{workspace} is a tool for writing and evaluating snippets of code. You can also use it to store arbitrary text.
  \item You can use \ind{keyboard shortcuts} on text in the workspace, or any other tool, to evaluate code. The most important of these are \menu{do it} (\short{d}), \menu{print it} (\short{p}), \menu{inspect it} (\short{i}) and \menu{explore it} (\short{I}).
  \item \sqmap is a tool for loading useful packages from the Internet.
  \item The \emph{system browser} is the main tool for browsing \sq code, and for developing new code.
  \item The \emph{test runner} is a tool for running unit tests. It also supports Test Driven Development.
\end{itemize}

%=================================================================
\ifx\wholebook\relax\else 
   \bibliographystyle{jurabib}
   \nobibliography{scg}
   \end{document}
\fi
%=================================================================

%%% Local Variables:
%%% coding: utf-8
%%% mode: latex
%%% TeX-master: t
%%% TeX-PDF-mode: t
%%% ispell-local-dictionary: "english"
%%% End:
