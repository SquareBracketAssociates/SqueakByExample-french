% $Author$ serge
% $Date$
% $Revision$
% relecture et synchro avec la version originale: martial boniou
% Fri Dec 14 14:05:59 CET 2007
% note: j'ai corrige l'url de telechargement/commande de SBE
% relecture: rene mages : Mon Dec 24 11:47:29 CET 2007
%=================================================================
\ifx\wholebook\relax\else
% --------------------------------------------
% Lulu:
	\documentclass[a4paper,10pt,twoside]{book}
	\usepackage[papersize={6in,9in}]{geometry}
	\input{../common.tex}
	\setboolean{lulu}{true}
% --------------------------------------------
% A4:
%	\documentclass[a4paper,11pt,twoside]{book}
%	\input{../common.tex}
%	\usepackage{a4wide}
% --------------------------------------------
    \graphicspath{{figures/} {../figures/}}
	\begin{document}
\fi
%=================================================================
\renewcommand{\nnbb}[2]{} % Disable editorial comments
\sloppy
%=================================================================
\chapter{Pr\'eface}\label{cha:intro}

%=================================================================
\section*{Qu'est ce que \sq?}

\sq est une impl\'ementation moderne, libre et compl\`ete du langage de programmation \st et de son environnement.

\sq est extr\^emement portable --- m\^eme sa machine virtuelle est \'egalement \'ecrite avec \st, ce qui facilite son d\'ebogage, son analyse et les modifications. \sq est le v\'ehicule de tout un ensemble de projets innovants, des applications multim\'edias et \'educatives aux environnements de d\'eveloppement pour le web.

%=================================================================
\section*{Qui devrait lire ce livre?}

Ce livre pr\'esente diff\'erents aspects de \sq, en commen\c{c}ant par les concepts de base et en poursuivant vers des sujets plus avanc\'es.

Ce livre ne vous apprendra pas \`a programmer. Le lecteur doit avoir quelques notions concernant les langages de programmation. Quelques connaissances sur la programmation objet seront utiles.

Ce livre introduit l'environnement de programmation, le langage et
les outils de \sq. Vous serez confront\'e \`a de nombreuses bonnes
pratiques de Smalltalk, mais l'accent sera mis plus particuli\`erement
sur les aspects techniques et non sur la conception orient\'ee
objet. Nous vous pr\'esenterons, autant que possible, une foule 
d'exemples (nous avons \'et\'e inspir\'e par l'excellent livre de Alex
Sharp sur Smalltalk\cite{Shar97a}).
\index{Sharp, Alex}

Il y a plusieurs autres livres sur \st disponibles gratuitement sur le web mais aucun d'entre eux ne se concentrent sur \sq. Voyez par exemple:
\url{stephane.ducasse.free.fr/FreeBooks.html}

\ifluluelse{}{\newpage} % layout hint
%=================================================================
\section*{Un petit conseil}

% http://www.surfscranton.com/architecture/KnightsPrinciples.htm

Ne soyez pas frustr\'e par des \'el\'ements de \st que vous ne comprenez pas imm\'ediatement.
Vous n'avez pas tout \`a conna\^itre!
Alan Knight exprime ce principe comme suit~\footnote{\url{www.surfscranton.com/architecture/KnightsPrinciples.htm}}:
\index{Knight, Alan}
\important{{\bf Ne vous en pr\'eoccupez pas!}%
%\important{{\bf Moquez-vous en!}%
~\footnote{Dans sa version originale: ``Try not to care''.}
Les d\'eveloppeurs \st d\'ebutants ont souvent beaucoup de
difficult\'es car ils pensent qu'il est n\'ecessaire de conna\^itre
tous les d\'etails d'une chose avant de l'utiliser. Cela signifie
qu'il leur faut un moment avant de ma\^{\i}triser un simple: \ct{Transcript show: 'Hello World'}. Une des grandes avanc\'ees de la programmation par objets est de pouvoir r\'epondre \`a la question ``Comment ceci marche?'' avec  ``Je ne m'en pr\'eoccupe pas''.}

%=================================================================
\section*{Un livre ouvert}

Ce livre est ouvert dans plusieurs sens:

\begin{itemize}

\item	Le contenu de ce livre est diffus\'e sous la licence Creative Commons Paternit\'e - Partage des Conditions Initiales \`a l'Identique.
		En r\'esum\'e, vous \^etes autoris\'e \`a partager librement et \`a adapter ce livre, tant que vous respectez les conditions de la licence disponible \`a l'adresse suivante: 
		\url{creativecommons.org/licenses/by-sa/3.0/}.

\item	Ce livre d\'ecrit simplement les concepts de base de \sq.
		Id\'ealement, nous voulons encourager de nouvelles personnes \`a contribuer \`a des chapitres sur des parties de \sq qui ne sont pas encore d\'ecrites.
		Si vous voulez participer \`a ce travail, merci de nous contacter. Nous voulons voir ce livre se d\'evelopper!
\end{itemize}

Plus de d\'etails concernant ce livre sont disponibles sur le site
web, \spe, h\'eberg\'e par l' \emph{Institute of Computer Science and
  Applied Mathematics} de l'Universit\'e de Berne en Suisse.

%=================================================================
\section*{La communaut\'e \sq}

La communaut\'e \sq est amicale et active.
Voici une courte liste de ressources que vous pourrez vous \^etre utile:

\begin{itemize}
\item \url{www.squeak.org} est le site web principal de \sq (\`a ne
  pas confondre avec  \url{www.squeakland.org} qui est consacr\'e \`a
  l'environnement EToy, construit au-dessus de \sq mais pour qui
  les enseignants d'\'ecoles \'el\'ementaires sont le public vis\'e).

\item \url{www.squeaksource.com}: \squeaksource est l'\'equivalent de \sourceforge pour les projets \sq.
\end{itemize}

\paragraph{\`A propos des listes de diffusion.} Il y a de nombreuses listes de diffusion qui sont parfois un petit trop active. Si vous ne voulez pas \^etre submerg\'e par les messages mais vous souhaitez n\'eanmoins participer nous vous conseillons d'utiliser \url{news.gmane.org} ou \url{www.nabble.com/Squeak-f14152.html} pour parcourir les listes.

Vous pouvez trouver l'ensemble complet des listes de diffusion de \sq \`a l'adresse \url{lists.squeakfoundation.org/mailman/listinfo}.

\begin{itemize}
\item Notez que \emph{\sq-dev} fait r\'ef\'erence \`a la liste de diffusion des d\'eveloppeurs  que l'on peut parcourir ici:\\
\url{news.gmane.org/gmane.comp.lang.smalltalk.squeak.general}
\item \emph{Newbies} fait r\'ef\'erence \`a une liste de diffusion
  pour les d\'ebutants o\`u les questions peuvent \^etre pos\'ees (il
  y a tant \`a apprendre que nous sommes tous des d\'ebutants d'un
  aspect de \sq!):\\
\url{news.gmane.org/gmane.comp.lang.smalltalk.squeak.beginners}
%note de martial: -- ajout -- IMPORTANT -- a confirmer
\item Les listes de diffusion pr\'ec\'edentes sont anglophones. La
  liste de diffusion francophone de \sq, \emph{Squeak-fr}, peut se
  lire \`a l'adresse:\\
\url{news.gmane.org/gmane.comp.lang.smalltalk.squeak.french}
\end{itemize}

\paragraph{IRC.}
Vous avez une question dont vous voulez trouver la r\'eponse
rapidement? Vous voulez rencontrer d'autres \emph{squeakers} de part
le monde? Un bon endroit pour participer \`a des discutions est le
canal IRC ``\#squeak'' sur \url{irc.freenode.net}. 
Venez y faire un tour pour dire ``Bonjour!''.

\paragraph{Autres sites.} Il y a de nombreux autres sites web supportant la communaut\'e \sq d'une mani\`ere ou d'une autre. En voici quelques-uns:
\begin{itemize}
  \item \url{people.squeakfoundation.org} est le site
    \textsf{SqueakPeople}, qui est une sorte de
    ``\url{advogato.org}'' pour les utilisateurs de \sq. Il offre des
    articles, des journaux et un syt\`eme int\'eressant de mesure de confiance.

  \item \url{planet.squeak.org} est le site de \textsf{PlanetSqueak},
    un agr\'egateur de fils RSS concernant \sq. C'est le bon endroit
    pour d\'ecouvrir des nouvelles sur \sq comme les derniers billets
    des blogs des d\'eveloppeurs et de tous ceux qui s'int\'eressent \`a \sq.

  \item \url{www.frappr.com/squeak} est un site qui r\'epertorie les
    diff\'erents utilisateurs de \sq dans le monde.

\end{itemize}

% Ajouter ici des sites spécifiques à la communauté francophone : 
% community.ofset.org/wiki/Squek
% planet-fr.squeak.org

%=================================================================
\section*{Exemples et exercices}

Nous utilisons deux conventions typographiques dans ce livre.

Nous avons essay\'e de fournir autant d'exemples que possible.
Il y a notamment plusieurs exemples avec des fragments de code qui
peuvent \^etre \'evalu\'es. Nous utilisons le symbole \ct{-->} afin
d'indiquer le r\'esultat qui peut \^etre obtenu en s\'electionnant
l'expression et en utilisant l'option \menu{print it} du menu contextuel:

\begin{code}{@TEST}
3 + 4 --> 7    "Si vous !s\'electionner! 3+4 et 'print it', 7 s'affichera"
\end{code}

% mise a jour (12/2007)
Si vous voulez d\'ecouvrir \sq en vous amusant avec ces morceaux de
code, sachez que vous pouvez charger un fichier texte avec la
totalit\'e des codes d'exemple via le site web du livre: \spe. 

La deuxi\`eme convention que nous utilisons est l'ic\^one
\dothisicon{} pour vous indiquer que vous avez quelque chose \`a faire: 

\dothis{Avancez et lisez le prochain chapitre!}

%=================================================================
%\section*{Typographic convention}

%\on{This is repeated in the First Application chapter.  I suggest we remove it from the Preface.}

%Programming in \st means defining classes and methods.
%Unlike most programming languages where programs sit in files, in \st classes and methods are objects too, and they are edited using a dedicated code browser.
%The browser will show you the code of a method in the context of the class it belongs to.

%Unfortunately this book is not (yet) interactive, so when we show you the code of a method, it is not always immediately clear for which class it is defined.
%For example, we cannot immediately tell which class the method \ct{cellsPerSide} belongs to:

%\begin{code}{}
%cellsPerSide
%   "The number of cells along each side of the game"
%   ^ 10
%\end{code}

%The \st convention to indicate that a method \ct{aMethod} belongs to a class \ct{aClass} is to write its name as \ct{aClass>>>aMethod}.
%So, if it is not immediately clear from the context which class a method belongs to, we will show it explicitly like this:

%\begin{code}{}
%SBEGame>>>cellsPerSide
%   "The number of cells along each side of the game"
%   ^ 10
%\end{code}

%Of course, when you actually type the code of the method into the browser, you don't have to type the class name or the \ct{>>>}; instead, you just make sure that the appropriate class is selected in the browser.

%=================================================================
\section*{Remerciements}

% We would like to thank various people who have contributed to this book.
% In particular, we thank

%martial: a reformuler - mettre 'les auteurs' a la place de nous
Nous voulons remercier Hilaire Fernandes et Serge Stinckwich qui nous
ont autoris\'e \`a traduire des parties de leurs articles sur \st et
Damien Cassou pour sa contribution au chapitre sur les flots de
donn\'ees ou \emph{streams}.
Nous voulons \'egalement remercier Tim Rowledge pour l'utilisation
du logo \sq et Frederica Nierstrasz pour le dessin de la couverture.

%  update (12/2007) - a relire
Nous remercions particuli\`erement Lukas Renggli et Orla Greevy pour leurs
commentaires sur les copies de la premi\`ere \'edition
%ajout
originale.

Nous remercions l'Universit\'e de Berne en Suisse pour le soutien
gracieusement offert \`a cette entreprise \emph{Open Source} et pour
les facilit\'es d'h\'ebergement web de ce livre.

% note sur les traducteurs ?

% note sur de la bibliographie en francais (Smalltalk + Squeak (Briffault, Bots...))

%=============================================================
\ifx\wholebook\relax\else
   \bibliographystyle{jurabib}
   \nobibliography{scg}
   \end{document}
\fi
%=============================================================
