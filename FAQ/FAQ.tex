% $Author: oscar $
% $Date: 2007-09-23 11:56:47 +0200 (dim, 23 sep 2007) $
% $Revision: 12130 $
% Traduction: Benoît TUDURI 18-10-2007
%% Relecture: Martial Boniou - Fri Nov  9  16:49:46 CET 2007
%% Relecture: Rene Mages     - Fri Jan  10 22:23:44 CET 2008
%=================================================================
\ifx\wholebook\relax\else
% --------------------------------------------
% Lulu:
	\documentclass[a4paper,10pt,twoside]{book}
	\usepackage[
		papersize={6in,9in},
		hmargin={.75in,.75in},
		vmargin={.75in,1in},
		ignoreheadfoot
	]{geometry}
	\input{../common.tex}
	\pagestyle{headings}
	\setboolean{lulu}{true}
% --------------------------------------------
% A4:
%	\documentclass[a4paper,11pt,twoside]{book}
%	\input{../common.tex}
%	\usepackage{a4wide}
% --------------------------------------------
    \graphicspath{{figures/} {../figures/}}
	\begin{document}
	\renewcommand{\nnbb}[2]{} % Disable editorial comments
	\sloppy
\fi
  
%=================================================================
%\chapter{Frequently Asked Questions}
%\chapter{Questions Fr\'equemments Pos\'ees}
%% note de Martial: le terme FAQ est utilise dans certaines pages; sinon "questions freq. posees" est valable aussi
\chapter{Foire Aux Questions}
\label{cha:faq}

%\on{These should be *real* (not invented) FAQs.
%Here are a few that I have collected.
%Check the ST lecture notes for more FAQs.
%We should also try to mine more from newbies.}
\on{Ceci devrait \^etre une *vraie* FAQs.
Celle-ci contient quelques questions que j'ai collect\'ees.
Pour plus de FAQ, vous pouvez consulter les r\'ef\'erences de lectures sur ST.
Nous devrions essayer de nous m\^emes pour les newbies...}

%=================================================================
%\section{Getting started}
\section{Pr\'emices}
\begin{faq}
%Where do I get the latest Squeak?
O\`u puis-je trouver la derni\`ere version de Squeak
\end{faq}
\answer
\url{ftp.squeak.org/current_development}
\index{download}

\begin{faq}
%Where is the Squeak ``Development Image''?
O\`u est ``l'Image de D\'eveloppement'' de Squeak~?
\end{faq}
\answer
%\url{www.squeaksource.com/ImageForDevelopers}
\squeakdev
%This is a specially-prepared image including lots of useful packages pre-installed for developers.
Ceci une image pr\'epar\'ee sp\'ecifiquement pour les d\'eveloppeurs. Elle contient de multiples 
%%paquets 
paquetages pr\'e-install\'es pour les d\'eveloppeurs.
%=================================================================
\section{Collections}

\begin{faq}
%How do I sort an \clsind{OrderedCollection}?
Comment puis-je trier une \clsind{OrderedCollection}~?

\end{faq}
\answer
%Send it the message \mthind{Collection}{asSortedCollection}.
Envoyez le message suivant \mthind{Collection}{asSortedCollection}.

\begin{code}{@TEST}
#(7 2 6 1) asSortedCollection --> a SortedCollection(1 2 6 7)
\end{code}

\begin{faq}
%How do I convert a collection of characters to a \clsind{String}?
Comment puis-je convertir une collection de caract\`eres en une 
%% ajout
cha\^ine de caract\`eres
\clsind{String}~?
\end{faq}
\answer
\begin{code}{@TEST}
String streamContents: [:str | str nextPutAll: 'hello' asSet] --> 'hleo'
\end{code}

%=================================================================
%\section{Browsing the system}
\section{Naviguer dans le syst\`eme}

\begin{faq}
%How to I search for a class?
Comment puis-je chercher une classe~?
\end{faq}
\answer
%\short{b} (browse) on the class name, or \short{f} in the category pane of the class browser.
\short{b} (pour \emph{browse} \cad parcourir \`a l'aide du navigateur de classes) via le nom de la classe ou \short{f} dans le panneau de navigation des cat\'egories de classes.
\index{keyboard shortcut!browse it}
\index{keyboard shortcut!find ...}

\begin{faq}
%How do I find/browse all sends to super?
Comment puis-je trouver/naviguer dans tous les envois 
%%de
\`a
 super~?
\end{faq}
\answer
%The second solution is much faster:
La deuxi\`eme solution est la plus rapide:
\begin{code}{}
SystemNavigation default browseMethodsWithSourceString: 'super'.
SystemNavigation default browseAllSelect: [:method | method sendsToSuper ].
\end{code}
%\index{browsing programmatically}
\index{naviguer de mani\`ere pragmatique}
\clsindex{SystemNavigation}

\begin{faq}
%How do I browse all super \subind{super}{send}{}s within a hierarchy?
Comment puis-je naviguer dans toute la hi\'erarchie des messages de toutes les super-classes  \subind{super}{send}{}~?
\end{faq}
\answer
\begin{code}{}
browseSuperSends:= [:aClass | SystemNavigation default
	browseMessageList: (aClass withAllSubclasses gather: [:each |
		(each methodDict associations
			select: [:assoc | assoc value sendsToSuper ])
				collect: [:assoc | MethodReference class: each selector: assoc key ] ])
	name: 'Supersends of ' , aClass name , ' and its subclasses'].
browseSuperSends value: OrderedCollection.
\end{code}

\begin{faq}
%How do I find out which are the new methods implemented in a class?
Comment puis-je d\'ecouvrir quelles sont les nouvelles m\'ethodes impl\'ement\'ees dans une classe?
\end{faq}
\answer
%Here we ask which new methods are introduced by \ct{True}:
Dans le cas pr\'esent nous demandons quelles sont les nouvelles m\'ethodes introduites par \ct{True}:
\begin{code}{@TEST | newMethods |}
newMethods:= [:aClass| aClass methodDict keys select:
	[:aMethod | (aClass superclass canUnderstand: aMethod) not ]].
newMethods value: True --> an IdentitySet(#asBit)
\end{code}

\begin{faq}
%How do I tell which methods of a class are abstract?
Comment puis-je 
%rapporter
trouver
les m\'ethodes d'une classe qui sont abstraites~?
\end{faq}
\answer
\begin{code}{@TEST | abstractMethods |}
abstractMethods:=
	[:aClass | aClass methodDict keys select:
		[:aMethod | (aClass>>aMethod) isAbstract ]].
abstractMethods value: Collection --> an IdentitySet(#remove:ifAbsent: #add: #do:)
\end{code}

\begin{faq}
%How do I generate a view of the \ind{AST} of an expression?
Comment puis-je cr\'eer une vue de 
%% ajout
l'arbre syntaxique abstrait ou 
\ind{AST} d'une expression~?
\end{faq}
\answer
%Load AST from squeaksource.com. Then evaluate:
Charger le paquetage AST depuis squeaksource.com. Ensuite \'evaluer:
\begin{code}{}
(RBParser parseExpression: '3+4') explore
\end{code}
%(Alternatively \emph{explore it}.)
(D'autre part \emph{explorer le}.)
\clsindex{RBParser}

\begin{faq}
%How do I find all the Traits in the system?
Comment puis-je trouver tout les \emph{traits} dans le syst\`eme~?
\end{faq}
\answer
\begin{code}{}
Smalltalk allTraits
\end{code}

\begin{faq}
%How do I find which classes use traits?
Comment puis-je trouver quelles classes utilisent les \emph{traits}~?
\end{faq}
\answer
\begin{code}{}
Smalltalk allClasses select: [:each | each hasTraitComposition ]
\end{code}

%=================================================================
%\section{Using Monticello and SqueakSource}
\section{Utilisation de Monticello et de SqueakSource}

\begin{faq}
%How do I load a \ind{Squeaksource} project?
Comment puis-je charger un projet du \ind{Squeaksource}~?
\index{Monticello}
\end{faq}
\answer
\begin{enumerate}
%  \item Find the project you want in \url{squeaksource.com}
  \item Trouvez le projet que vous souhaitez sur \url{squeaksource.com}
%  \item Copy the registration code snippet
  \item Copiez le code d'enregistrement
%  \item Select \menu{open \go Monticello browser}
  \item S\'electionnez \menu{open \go Monticello browser}
%  \item Select \menu{+Repository \go HTTP}
  \item S\'electionnez \menu{+Repository \go HTTP}
%  \item Paste and accept the Registration code snippet; enter your password
  \item Collez et acceptez le code d'enregistrement ; entrez votre mot de passe
%  \item Select the new repository and \menu{Open} it
  \item S\'electionnez le nouveau d\'ep\^ot et ouvrez-le avec le bouton \menu{Open}
%  \item Select and load the latest version
  \item S\'electionnez et chargez la version la plus r\'ecente
\end{enumerate}

\begin{faq}
%How do I create a SqueakSource project?
Comment puis-je cr\'eer un projet SqueakSource~?
\end{faq}
\answer
\begin{enumerate}
%  \item Go to \url{squeaksource.com}
  \item Allez à \url{squeaksource.com}
%  \item Register yourself as a new member
  \item Enregistrez-vous comme un nouveau membre
%  \item Register a project (name = category)
  \item Enregistrez un projet (nom = cat\'egorie)
%  \item Copy the Registration code snippet
  \item Copiez le code d'enregistrement
%  \item \menu{open \go Monticello browser}
  \item \menu{open \go Monticello browser}
%  \item \menu{+Package} to add the category
  \item \menu{+Package} pour ajouter une cat\'egorie
%  \item Select the package
  \item S\'electionnez le package
%  \item \menu{+Repository \go HTTP}
  \item \menu{+Repository \go HTTP}
%  \item Paste and accept the Registration code snippet; enter your password
  \item Collez et acceptez le code d'enregistrement ; entrez votre mot de passe
%  \item \menu{Save} to save the first version
  \item \menu{Save} pour enregistrer la premi\`ere version
\end{enumerate}

\begin{faq}
%How do I extend \ct{Number} with \ct{Number>>>chf} but have Monticello recognize it as being part of my \ct{Money} project?
Comment puis-je \'etendre \ct{Number} avec
%% ajout
la m\'ethode \ct{Number>>>chf} 
%% changement
tel que Monticello la reconnaissent comme \'etant une partie de mon projet \ct{Money}~?
\end{faq}
\answer
%Put it in a method-category named \ct{*Money}.
%% changement
Mettez-la 
dans une cat\'egorie de m\'ethodes nomm\'ee \ct{*Money}.
%Monticello gathers all methods that are in other categories named like *package and includes them in your package.
Monticello r\'eunit toutes les m\'ethodes 
dont les noms de cat\'egories 
%% changement: nomm\'ees autrement telles que 
ont la forme \emph{*package} et les ins\'ere dans votre package.

%=================================================================
%\section{Tools}
\section{Outils}

\begin{faq}
%How do I programmatically open the \ind{SUnit} \ind{TestRunner}?
Comment puis-je ouvrir de mani\`ere pragmatique le \ind{SUnit} \ind{TestRunner}~?
\end{faq}
\answer
%Evaluate \ct{TestRunner open}.
\'Evaluez \ct{TestRunner open}.

\begin{faq}
%Where can I find the \ind{Refactoring Browser}?
O\`u puis-je trouver le \ind{Refactoring Browser}~?
\end{faq}
\answer
%Load AST then Refactoring Engine from squeaksource.com:
Chargez le paquetage AST puis le moteur de 
%%plutot que refrabrication 
refactorisation sur le site squeaksource.com:\\
\url{www.squeaksource.com/AST}\\
\url{www.squeaksource.com/RefactoringEngine}

\begin{faq}
%How do I register the browser that I want to be the default?
Comment puis-je enregistrer le navigateur comme navigateur par d\'efaut~?
\end{faq}
\answer
%Click the menu icon in the top left of the Browser window.
Cliquez sur l'ic\^one du menu situ\'e en haut à gauche de la fen\^etre Browser \`a c\^ot\'e de la croix de destruction de la fen\^etre. 
%% ajout (pour la traduction)
Choisissez \menu{Register this Browser as default} pour enregistrer le navigateur courant comme navigateur par d\'efaut ou bien, s\'electionnez \menu{Choose new default Browser} pour obtenir un menu flottant d'o\`u vous pourrez faire votre choix parmi les diff\'erentes classes de Browser.

%=================================================================
%\section{Regular expressions and parsing}
\section{Expressions r\'eguli\`eres et analyse grammaticale}

\begin{faq}
%How can I work with regular expressions?
Comment puis-je travailler avec les expressions r\'eguli\`eres~?
\index{regular expression package}
\end{faq}
\answer
%\ifluluelse
%	{Load Vassili Bykov's RegEx package from: \\
%	{
Chargez le paquetage de RegEx de Vassili Bykov \`a l'adresse: \\
\url{www.squeaksource.com/Regex.html}
%}
%	{Load Vassili Bykov's RegEx package from:
%	{Chargez le paquet de RegEx Vassili Bykov's RegEx sur:
%	\url{www.squeaksource.com/Regex.html}}
\index{Bykov, Vassili}

\begin{faq}
%Where is the documentation for the RegEx package?
O\`u est la documentation pour le paquetage RegEx~?
\end{faq}
\answer
%Look at the \menu{DOCUMENTATION} protocol of \ct{RxParser class} in the \menu{VB-Regex} category.
%% changement
Regardez dans le protocole \menu{DOCUMENTATION} de \ct{RxParser class} situ\'e dans la cat\'egorie \menu{VB-Regex}.

\begin{faq}
%Are there tools for writing parsers?
Y a-t'il des outils pour l'\'ecriture d'un outil d'analyse grammaticale~?
\end{faq}
\answer
%Use \ind{SmaCC}\,---\,the Smalltalk Compiler Compiler.
Utilisez \ind{SmaCC}\,---\,le compilateur de compilateur (ou g\'en\'erateur de compilateur)~\footnote{En anglais, Compiler-Compiler.} Smalltalk.
%You should install at least SmaCC-lr.13.
Vous devrez installer au moins SmaCC-lr.13.
%Load it from \url{www.squeaksource.com/SmaccDevelopment.html}.
Chargez-le depuis \url{www.squeaksource.com/SmaccDevelopment.html}.
%There is a nice tutorial online:
Il y a un bon tutoriel en ligne \`a l'adresse:
\url{www.refactory.com/Software/SmaCC/Tutorial.html}

\begin{faq}
%Which packages should I load from SqueakSource SmaccDevelopment to write parsers?
Quels paquetages devrais-je charger depuis \emph{SqueakSource SmaccDevelopment} pour \'ecrire un analyseur grammatical~?
\end{faq}
\answer
%Load the latest version of \ind{SmaCCDev}{}\,---\,the runtime is already there.
Chargez la derni\`ere version de \ind{SmaCCDev}{}\,---\,le lanceur de programme est d\'ej\`a actif.
(Attention: SmaCC-Development est destin\'e \`a la version 3.8 de \sq)

%=================================================================
\ifx\wholebook\relax\else\end{document}\fi
%=================================================================

%\begin{faq}
%How do I run a headless image with a file argument?
%\end{faq}
%\answer
%Right now you can't do it with the MacOSX VM.

%\begin{faq}
%How do I find out which methods access the Smalltalk dictionary?
%\end{faq}
%\answer
%???

%\begin{faq}
%How do I get the tree view of an AST?
%\end{faq}
%\answer
%???

%\begin{faq}
%How do I browse all references to a given class?
%\end{faq}
%\answer
%???

%%%%%%%%%%%%%%%%%%%%%%%%%%%%%%%%%%%%%

%  Benoît

% Rappel LaTeX:
%  é    \'e
%  è    \`e
%  ê    \^e
%  ë    \"e
%  ç    \c{c}
%  $    \$
%  &    \&
%  %    \%
%  #    \#
%  _    \_
%  {    \{
%  }    \}
%  ~    \~
%  ^    \^
%  \    $\backslash$

% Paramètre VIM 7.1
% pour l'édition de ce document
% :set encoding=UTF-8
%%%%%%%%%%%%%%%%%%%%%%%%%%%%%%%%%%%%%

